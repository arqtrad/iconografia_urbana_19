\hypertarget{introduuxe7uxe3o}{%
\section{Introdução}\label{introduuxe7uxe3o}}

No Brasil de meados do século XIX, a pintura paisagística e a literatura
crítica ou cronística apresentam a cidade e a arquitetura como
instrumentos privilegiados na articulação de identidades e na construção
antropológica da nacionalidade, construção esta que é característica das
periferias pós-coloniais \autocite[173]{stocking:1982afterword47}. O
processo pelo qual o imaginário pictórico e verbal elabora a sua própria
realidade discursiva no Brasil oitocentista é bem conhecido
\autocite[8]{pesavento:2002imaginario}. No entanto, a interpretação
histórica hegemônica tem apresentado o período imperial como época de
\emph{afirmação}, ainda que paulatina, da identidade nacional
\autocite{wehling:1983origens338,moreira:2003historia43}, ao passo que
tem reservado à primeira República o papel de \emph{problematizar} a
questão nacional \autocite{oliveira:1990questao,bernd:1992literatura}.
Essa interpretação deixa em segundo plano, porém, as raízes da
problematização da identidade nacional ainda durante o Império. Tais
raízes emergem na forma de uma teorização do caráter artístico
brasileiro como uma lacuna a ser preenchida na construção da
nacionalidade.

De fato, nas representações pictóricas assim como na crítica artística e
arquitetônica de meados do século XIX a início do XX, a busca por um
\emph{cerne} da identidade nacional na pintura e na arquitetura
desemboca frequentemente na conceituação da produção artística como uma
\emph{ausência} de ``brasilidade''. Tal ausência é, todavia, o resultado
necessário das próprias premissas teóricas dessa busca pela identidade
brasileira: uma lacuna intencional, a ser preenchida pelo conceito
prescritivo de caráter nacional formulado no discurso crítico.

A problematização da identidade brasileira por meio da arte e da
representação pictórica do espaço urbano se desenrola, então, segundo a
lógica própria ao desenvolvimento da crítica de arte como gênero
literário no Brasil, bem como à profissionalização da disciplina
arquitetônica. A formulação discursiva do caráter artístico nacional,
nesse âmbito, faz pouco caso da periodização convencional da história
política, sendo melhor divida em três fases distintas:

\begin{enumerate}
\def\labelenumi{(\arabic{enumi})}
\tightlist
\item
  Da Independência até a década de 1870, a crítica e a historiografia
  assumem o papel de afirmação de uma identidade nacional, ainda que
  prospectiva, mas na produção pictórica se percebem os primeiros sinais
  de uma inquietação com o ambiente cultural brasileiro;
\item
  Da década de 1870 até a de 1910, o caráter nacional é posto claramente
  como um problema artístico a ser resolvido no futuro --- resolução que
  passará, por influência da crítica literária coetânea,
  predominantemente pela questão dos temas e objetos da produção
  artística;
\item
  Da década de 1910 até o início da década de 1940, a crescente inserção
  dos arquitetos no universo do discurso erudito e político leva à
  projeção do problema do caráter nacional não mais no futuro, e sim no
  passado das cidades históricas, deslocando também o foco do problema
  da \emph{temática} da obra de arte para o seu \emph{estilo}.
\end{enumerate}
