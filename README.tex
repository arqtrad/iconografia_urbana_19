\hypertarget{a-cidade-na-arte-e-a-arte-na-cidade-vistas-urbanas-no-brasil-do-suxe9culo-xix}{%
\section{A cidade na arte e a arte na cidade: Vistas urbanas no Brasil
do século
XIX}\label{a-cidade-na-arte-e-a-arte-na-cidade-vistas-urbanas-no-brasil-do-suxe9culo-xix}}

\begin{fignos:no-prefix-figure-caption}

\begin{figure}
\centering
\includegraphics{https://github.com/dmcpatrimonio/iconografia_urbana_19/workflows/Authorea/badge.svg}
\caption{Authorea}
\end{figure}

\end{fignos:no-prefix-figure-caption}

Texto completo disponível nos anais do IV seminário internacional das
Academias de Arquitetura e Urbanismo de Língua Portuguesa,
\href{http://aeaulp.com/alinguaquehabitamos/index.php/pt/publicacao}{\emph{A
língua que habitamos}, v. 5}

Durante o século XIX, artistas e viajantes estavam intensamente
explorando o contexto natural e cultural do Brasil. Relatos de
exploradores estrangeiros, a pintura acadêmica e os primórdios da
fotografia eram instrumentos privilegiados para se transmitir a
aparência visível e, principalmente, construir um discurso sobre o
caráter moral e histórico da nação. Esse discurso, pautado, segundo
Fedatto, pelo princípio de uma ``singularidade brasileira'', oscila
entre a afirmação de uma identidade nacional fixa e o conformismo diante
da natureza sobrepujando a civilização.

Ambas abordagens se refletem nas pacatas vistas urbanas produzidas pelos
artistas brasileiros e estrangeiros ao longo da maior parte do século
XIX. Em contraste com as academias europeias, defensoras da hierarquia
dos gêneros de pintura, o contexto artístico brasileiro era de
predomínio da pintura de gênero e, secundariamente, da paisagem. Esses
dois domínios se confundem na representação de cenas urbanas, que gozam,
no Brasil, de papel informativo e estético peculiar.

Como uma espécie de \emph{exotismo para dentro}, essas obras, expostas
nos Salões de Belas Artes ou compondo coleções particulares, tecem um
discurso político e etnográfico sobre a cidade real vivenciada pelo
artista e a civilização urbana por ele imaginada. A literatura do início
do século XX, programaticamente comprometida com um projeto de
modernização cultural, teve relativo sucesso em relegar essa figuração
da civilização oitocentista ao papel de modorrento retrato do atraso
nacional. A presente pesquisa desconstrói essa imagem, apontando o
\emph{locus} ideológico e estético da vista urbana no entendimento do
que seja a natureza da cidade brasileira.

De um lado, afirma-se o papel político que tem a vista urbana na
elaboração de uma ordem administrativa centralista e hierarquizada ---
evidenciada pela predominância de panoramas de cidades administrativas
e, dentro dessas, de palácios governamentais ou igrejas associadas ao
poder imperial. Do outro, resgata-se a categoria estética do sublime
como contraponto ao totalitarismo da mensagem política. Nesse registro,
a cidade brasileira é apropriada enquanto intromissão de um ideal
civilizador na onipotência da Natureza, percebida, em última análise,
como o elemento definidor do Brasil.

Trata-se, ao mesmo tempo, de uma visão que aplasta as diferenças
políticas, sociais e regionais. A cidade é apresentada sob o prisma de
sua unidade num panorama distante ou na representatividade de um
monumento, ao passo que os conflitos sociais e a identidade dos lugares
--- capturados pelos viajantes estrangeiros do início do século XIX ---
desaparecem temporariamente do registro visual. Paradoxalmente, é com a
influência do impressionismo e do realismo na pintura brasileira que se
perde essa dupla leitura do projeto civilizador e de sua limitação
diante do sublime. Nesse momento, ao final do século XIX, as
perspectivas se restringem, a fotografia documenta os estertores de um
regime político, da instituição escravista e de bairros inteiros
apagados pela marcha do progresso republicano.
