\hypertarget{rise-and-fall-of-tradition}{%
\section{Rise and Fall of Tradition}\label{rise-and-fall-of-tradition}}

Monteiro Lobato saw in material and cultural modernization the only
means of escape from this eternal decay, thus using the heyday of the
coffee-growing urban society as little more than an abstract backdrop to
his criticism of contemporary backwardness. Meanwhile, similar ideas
about a stifled development of national character, followed by a period
of decay, and the need to reassert the greatness of Brazilian identity,
were being promoted in São Paulo by a Portuguese engineer, archaeologist
and political activist, Ricardo Severo. His presence in Brazil coincided
with the development of traditionalist movements throughout the Americas
in the early twentieth century. Rooted in the European romantic
nationalism of the mid-nineteenth century, traditionalist architecture
in the Western Hemisphere took root first in the United States, where
the Mission Style provided a template for Hispanic revivals. It then
spread southward, reaching Latin America around the time of the
centennial of independence in many of its countries
\autocite[p.~12]{amaral:1994invencion}

\begin{center}\rule{0.5\linewidth}{0.5pt}\end{center}
