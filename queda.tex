\hypertarget{tradiuxe7uxe3o-e-queda}{%
\section{Tradição e Queda}\label{tradiuxe7uxe3o-e-queda}}

Os anos de crescimento econômico e efervescência cultural entre a
renegociação da dívida externa (1902) e a Exposição do Centenário da
abertura dos portos foram palco da articulação gradual de interesses
conflitantes: cosmopolitismo e expressão de um caráter nacional,
exaltação da modernidade e da industrialização juntamente com um gosto
pela exuberância da natureza tropical, unidade nacional e exibição das
particularidades estaduais \autocite{pereira:2011exposicao}. Nesse
contexto, um caráter nacional na arte era mais um desejo otimista do que
algo já realizado em algum momento do passado ou no presente. A
problemática da influência portuguesa e da emancipação cultural do
Brasil reforçava a diferença com respeito ao romantismo europeu, o qual
considerava o caráter nacional como já realizado no passado e buscava
reatualizá-lo.

Essa colocação do caráter brasileiro num futuro mais ou menos próximo
foi favoravelmente acolhida por autores e movimentos que buscavam
apropriar-se da tarefa de construir uma \emph{brasilidade} no primeiro
quartel do século XX. Não obstante os indianismos literário e artístico
da segunda metade do século anterior, considerava-se que nenhum
movimento pregresso havia logrado a expressão de um caráter nacional. A
tarefa estava, portanto, aberta aos contemporâneos, alguns dos quais não
pretendiam esperar pacientemente, como Gonzaga Duque, a coalescência
natural desse caráter, mas avançavam um programa para criá-lo.

Monteiro Lobato, já reconhecido jornalista e crítico, irrompeu na cena
literária com a coletânea \emph{Urupês} (1918), apresentando com
sarcasmo a tensão entre o desejo de modernidade e cosmopolitismo e o
atraso da sociedade agrária em São Paulo. O último capítulo do livro, em
particular, criticava o fascínio dos escritores com personagens
autenticamente nacionais, especialmente o caipira que se sobrepusera ao
índio como o brasileiro por excelência
\autocite[p.~208--209]{monteirolobato:1944urupes}. Assim como seus
contemporâneos paulistas, Monteiro Lobato era um entusiasta do progresso
material e da industrialização. Reciprocamente, sua prosa, ainda que
inspirada no naturalismo, apresentava uma justaposição sincopada
bastante moderna para a época. Também distanciava-se do naturalismo por
recusar a idealização do caipira enquanto representante autêntico do
caráter nacional \autocite[p.~303]{silva:2013modernidade}.

Sua crítica à inércia social e material das pequenas cidades
interioranas em \emph{Cidades Mortas} (1919) é mais complexa. A
autossuficiência pretensiosa da burguesia de Oblivion, por exemplo,
aludia a uma espécia de decrepitude intemporal, mais do que simples
subdesenvolvimento. Os cidadãos de Oblivion por pouco não morriam de
tédio \autocite[p.~25]{monteirolobato:1919cidades}, mas não de privação
material. As cidades mortas continham:

\begin{quote}
Impressões de uma juventude morta que vegetava na estagnação das cidades
mortas. Também há alguma coisa moderna. Mas tanto o moderno quanto o
antigo valem o mesmo --- nada.
\autocite[epígrafe]{monteirolobato:1919cidades}
\end{quote}

Esse senso de desesperança, de algo que teria sido mas não foi, estaria
muito claro na mente dos contemporâneos de Monteiro Lobato, claramente
referindo-se à decadência material e espiritual de centros urbanos antes
pujantes, nas regiões cafeeiras
\autocite[p.~299]{silva:2013modernidade}:

\begin{quote}
Ali, tudo era, nada é. Nenhum verbo é conjugado no presente. Tudo é
pretérito.

Um grupo de cidades moribundas arrastam uma vida decrépita, passada
chorando na miséria de hoje a grandeza nostálgica de antanho.
\autocite[p.~8]{monteirolobato:1919cidades}
\end{quote}

A antiga prosperidade dessas cidades nunca era mencionada diretamente no
livro, mas estaria vívida nas mentes dos seus leitores. Monteiro Lobato
rejeitava o ideal romântico e naturalista da glorificação de um caráter
nacional, ou mesmo regional, representado pela sociedade tradicional.
Entretanto, \emph{Cidades Mortas} apresenta outro tipo de nostalgia,
remetendo aos escritos de juventude de Gonzaga Duque: a noção de que
algum processo de germinação de um caráter nacional estivera gestando,
rude porém autêntico, e de que esse processo entrou em decadência antes
de dar frutos. A construção de um ``tempo fora do tempo'' mítico
\autocite{eliade:1966aspects}, de eterna decadência desde um cume que
nunca foi atingido, deu a tônica para a construção de uma imagem
altamente flexível de ``brasilidade'' perdida. Era crucial, para essa
retórica de uma perda mítica de algo que nunca existiu de fato,
vislumbrar um hiato cronológico implícito que mantivesse algum senso de
tempo histórico \autocite[p.~293]{silva:2012cidades}. Esse hiato também
estaria claro para o leitor contemporâneo, dispensando menção explícita
no texto: correspondia ao declínio do cultivo do café no Vale do
Paraíba, iniciado na década de 1860 e exacerbado a partir dos anos 1880.

\begin{center}\rule{0.5\linewidth}{0.5pt}\end{center}

Monteiro Lobato saw in material and cultural modernization the only
means of escape from this eternal decay, thus using the heyday of the
coffee-growing urban society as little more than an abstract backdrop to
his criticism of contemporary backwardness. Meanwhile, similar ideas
about a stifled development of national character, followed by a period
of decay, and the need to reassert the greatness of Brazilian identity,
were being promoted in São Paulo by a Portuguese engineer, archaeologist
and political activist, Ricardo Severo. His presence in Brazil coincided
with the development of traditionalist movements throughout the Americas
in the early twentieth century. Rooted in the European romantic
nationalism of the mid-nineteenth century, traditionalist architecture
in the Western Hemisphere took root first in the United States, where
the Mission Style provided a template for Hispanic revivals. It then
spread southward, reaching Latin America around the time of the
centennial of independence in many of its countries
\autocite[p.~12]{amaral:1994invencion}

\begin{center}\rule{0.5\linewidth}{0.5pt}\end{center}
