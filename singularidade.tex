\hypertarget{a-busca-pela-singularidade-do-caruxe1ter-brasileiro-na-paisagem}{%
\subsection{A busca pela singularidade do caráter brasileiro na
paisagem}\label{a-busca-pela-singularidade-do-caruxe1ter-brasileiro-na-paisagem}}

A construção de um discurso sobre o caráter moral e histórico da nação,
no século XIX, implica distinguir os aspectos peculiares na geografia,
nos costumes e na organização política de uma sociedade. É o que faz,
por exemplo, o pintor Jean-Baptiste Debret (1768--1848), um dos
protagonistas da Missão francesa que tanta polêmica suscitará mais
adiante, ao propor como objeto da sua publicação \emph{Voyage
pittoresque et historique au Brésil} os ``pontos característicos dos
objetos circundantes'' para assim constituir uma ``obra histórica''
\autocite[p.~i--ii]{debret:1834voyage1}. Para tanto, Debret atribui
papel preponderante ao seu ofício --- o desenho ---, mas não prescinde
de textos explicativos, especialmente necessários para tornar as cenas
do cotidiano escravista compreensíveis para o público francês.

Representar a aparência visível da cidade, especialmente com o reforço
da cena anedótica ou da curiosidade repulsiva, cumpre então seu papel de
instrumento privilegiado para caracterizar a fisionomia, por assim
dizer, do corpo social. Assim fazendo, Debret insere-se numa tradição de
arte metafórica, extraída da literatura no século XVII
\autocite{delft:1993nature}, codificada por Charles Le Brun
\autocite{dabbs:2002characterising65} e introduzida no domínio da
arquitetura e do desenho urbano com o neoclassicismo ``visionário'' da
idade do Iluminismo \autocite{pigafetta:2009passioni}. A
\emph{caracteriologia} classicista se constitui, assim, sobre a visão de
mundo cartesiana de uma natureza --- inclusive humana --- imutável,
arcabouço sobre o qual erigem-se as particularidades nacionais e
individuais.

Na deriva, apontada por Carolina Fedatto
\autocite*[102]{fedatto:2013saber}, do conceito sociológico de
``singularidade brasileira'' para os estudos de história cultural
urbana,

No longo caminho trilhado pelos ``estudos brasileiros'' desde então, a
simples exposição das particularidades nacionais tem degenerado não
tanto no rumo dos nacionalismos militantes que levaram a Europa a duas
guerras mundiais, mas principalmente na formulação de um princípio de
``singularidade brasileira'' a alicerçar as teses de autores do século
XX, tais como Gilberto Freyre, Sergio Buarque de Holanda, Raymundo Faoro
e Jessé Souza. Sem entrar no mérito desses seminais marcos explicativos
do Brasil no campo sociológico, observa-se que eles se situam no já
citado espectro crítico que busca desvencilhar a cultura latinoamericana
de uma dependência histórica com respeito aos marcos explicativos da
cultura europeia.

Na hierarquia acadêmica dos gêneros pictóricos, os temas religiosos e
mitológicos --- estes compreendendo também a história antiga --- são
aptos a expressarem os caracteres tidos como universais. À história
moderna, por sua vez, correspondem os caracteres nacionais. Estes
gêneros ocupam o patamar superior, predominantemente preenchido pela
encomenda da Igreja, do Estado, ou dos agentes privados da elite
política.

Esta hierarquia se rebate no influente tomo ``Arquitetura'' da
Enciclopédia de Diderot e d'Alembert redigido por Quatremère de Quincy.
No verbete ``Caráter'' \autocite*{quatremere:1788caractere}, Quatremère
estabelece um gradiente expressivo que começa com os caracteres
``essenciais'' dos princípios construtivos, passando pelas respostas
``relativas'', ou seja, nacionais a condicionantes geográficas,
climáticas e culturais, e terminando nas características ``imitativas'',
que consistem em individualizar um edifício segundo sua função ou outros
requisitos particularizados.

No entanto, em contraste com as academias europeias, defensoras da
hierarquia dos gêneros de pintura, o contexto artístico brasileiro no
século XIX é de predomínio da pintura de gênero e, secundariamente, da
paisagem \autocite{squeff:2012galeria}. Esses dois domínios se confundem
na representação de cenas urbanas, que gozam, no Brasil, de papel
informativo e estético peculiar. Como uma espécie de \emph{exotismo para
dentro}, essas obras, expostas nos Salões de Belas Artes ou compondo
coleções particulares, tecem um discurso político e etnográfico sobre a
cidade real vivenciada pelo artista e a civilização urbana por ele
imaginada. A literatura do início do século XX, programaticamente
comprometida com um projeto de modernização cultural, tem relativo
sucesso em relegar essa figuração da civilização oitocentista ao papel
de modorrento retrato do atraso nacional
\autocite{bernd:1992literatura}.
