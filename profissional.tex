\hypertarget{tradition-and-the-professional-architect}{%
\section{Tradition and the Professional
Architect}\label{tradition-and-the-professional-architect}}

The onset of the Modern movement in Brazilian architecture thus entailed
a power struggle between the proponents of the neocolonial movement and
their younger rivals. Both camps deployed the same narrative regarding
the development of national character in order to promote opposing views
of architectural style and space, and both derided their rival as being
so beneath them, it was ``non-architecture''. In the meantime, a single
dissonant chord struck the debate in Rio during that time. It was Adolfo
Morales de los Ríos Filho's book \emph{Grandjean de Montigny e a
evolução da arte brasileira} (Grandjean de Montigny and the Evolution of
Brazilian Art, 1941). In this work, Morales de los Ríos (1887--1973),
director of the National Fine Arts School,went back to the later Gonzaga
Duque's positive view of early-nineteenth-century French influence on
Brazilian art. The argument was similar:

\begin{quote}
Yes, it dignified Brazilian art, fighting the neglect and ignorance of a
fledgling society, .~.~.~. and contributing to the foundation of an art
school, where it would have been difficult to create it using existing
{[}local{]} resources. \autocite[p.~157]{morales:1941grandjean}
\end{quote}

Despite being part of the architectural establishment, as director of
the most important fine arts school in the nation, Morales de los Ríos
had his own axe to grind as well. The Beaux-Arts method had been under
critical fire for well over two decades, first from the traditionalist
movement, then from the modernists. The National Fine Arts School was
somewhat open, nevertheless, to the teaching of neocolonial architecture
in the 1920s, although it was considered but one of several eclectic
styles happily used and mixed by students and teachers. Lucio Costa
himself, owing to political connections, had seized the school
directorship for a few months in 1931, before being ousted by the
professors. Moreover, urban renewal in Rio, exacerbated since 1920, was
threatening the nineteenth-century French-inspired heritage just as much
as the monuments of the colonial period.

Morales de los Ríos's arguments, however, were of an obviously different
nature that those of the neocolonial-modernist groups. Unlike the
neocolonial or modernist mavericks, he was directly implicated in the
education of a class of elite artists expected to succeed in both public
and private commissions. Thus, he defended not only the historical roots
of his school, but also the diversity and adaptability of architects in
a time of rapidly changing tastes among the public, particularly so in a
moment when support for the neocolonial style in major works had all but
disappeared. Also, the Pan-American ideology of the first three decades
of the twentieth century, which had underwritten neocolonial
architecture throughout the continent, had been replaced by introverted
populism in Brazil's fascist government led by Getúlio Vargas.
Paradoxically, this introversion was deleterious for the traditional art
movement, as it made Brazilians look away from the art of the American
continent and towards the more conventional artistic centers of Europe.
Public architecture in the Vargas régime oscillated between the stripped
classicism then popular in most European countries, and modernism, which
was being half-heartedly supported by fascist Italy at the same time. As
for the fickle bourgeois of São Paulo, they moved on to favor variations
on Art Déco, Italian rationalism, and whitewashed modernism.

Both Ricardo Severo and José Mariano Filho, on the other hand, had
advocated a sort of sociological collectivism in the architectural
profession. Severo, a republican activist who at first moved to Brazil
to avoid political persecution, expected architecture and architects to
play a role in the forging of a modern---meaning nationalist---state,
conscious and proud of its ethnic origins
\autocite[p.~29]{mello:2007ricardo} He sought to balance his
archaeological interests, which led him to favor a structuralist
cohesion of sorts between a centuries-old culture and its present
developments, and his practice as an architect, where he ultimately gave
in to the public expectations of wholly modernized, eclectic plans and
picturesque massing. Nevertheless, he was successful in fostering public
taste for such traditional Brazilian elements as the seventeenth-century
\emph{alpendres} (deep and wide colonnaded porches) and generous roof
overhangs. These features went on to become favorites of Brazilian
single-family houses throughout the twentieth century.

As for Mariano, a scion of the landed elite of the Brazilian Northeast,
architecture was a dilettante passion as much as a political cause. Free
from the need to make a living out of the profession, he was thus little
interested in matters of professional cohesion and construction
industry. Conversely, with his disposable income, he was able to fund a
considerable documentation effort, as well as publicity stunts in the
form of design competitions. In addition to this, he was a regular
contributor to the Rio press throughout the 1930s. As the prestige of
neocolonial architecture for major public works eroded during that
decade, his criticism of the Modern movement increasingly resorted to
the sort of racial and political slander expected to appeal to the heads
of the fascist government: ``architectural Judaism'' and ``communist
architecture'' were expressions used in his later writings,
\autocite[p.~41]{mariannofilho:1943margem} as well as attacks on
artistic ``freemasonry.''

Both because of his early years in the neocolonial movement, and in
reaction to Mariano's criticism of modern architecture, Lucio Costa,
too, resorted to an ethnic narrative regarding the roots of national
artistic character. As a pure-bred white Brazilian of colonial
Portuguese descent, the son of a Navy officer, he was in as strong a
position as Mariano to claim authority to speak for national roots.
Moreover, his political connections in the Vargas government freed him
from the concern with day-to-day professional practice in a market
environment. Costa at first supported Mariano's narrative of a
collective, anonymous architecture without architects, even through his
first decade as a leader of the Modern movement in Brazil. This led him
to shun at first the few known masters of Brazilian art in the colonial
period. By 1945, however, his writings focused chiefly on self-conscious
artistic intent and the importance of individual genius for the
development of style. A hinge moment in his views probably occurred
around 1939, when he supported Oscar Niemeyer's attempt to insert a
modernist hotel at the heart of the historic district in Ouro Preto,
although influence from poets Mário de Andrade and Carlos Drummond de
Andrade in the SPHAN is not to be excluded. Costa then moved away from
the ethnological understanding of architectural coherence, to argue that
an architectural work of art ``shall not resent the proximity to other
works of art'' \autocite[quoted in][]{comas:2010passado} Throughout the
remainder of his long writing career, he strove to reconcile both views
as the discourses on the artistic originality of the Modern movement
became hegemonic. The unchallenged ethos of national genius that Costa
helped construct for Niemeyer remains to this day a favorite topic of
debate regarding the nature of professional practice in Brazilian
architecture.
