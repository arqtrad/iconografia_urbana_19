\hypertarget{introduction}{%
\section{Introduction}\label{introduction}}

Despite their ideological oppositions, Brazilian modernists and eclectic
nationalists in the late nineteenth and early twentieth century had one
stance in common: both groups agreed that the country's art and
architecture since the second half of the previous century lacked
national character and adaptation to Brazil's climate and social
conditions. This postulate was partly refuted in Portuguese-language
scholarship published since the 1960s, exposing the persistence of
colonial-era patterns in the hinterland and, in a few cases, in urban
settings. In the urge to rehabilitate nineteenth-century Brazilian art
and architecture, however, the actual discourses by which it came to be
ostracized were themselves suppressed from scholarship.

This paper examines a few landmark narratives on the issue of national
character published between 1880 and 1940. Some major authors in point
were academic art critic Gonzaga Duque, neocolonial engineer Ricardo
Severo and physician José Mariano Filho, Beaux-Arts architect Adolfo
Morales de los Ríos Filho, writer Monteiro Lobato, and modernist
architect Lucio Costa. These authors' writings will be examined here
with regard to their definition of a Brazilian character and the
purported lack thereof in works produced in the generations that
preceded them. The discourse on the lack of national character put
forward in these narratives stems both from well-documented aesthetic
agendas advanced by these authors, and from the less frequently
acknowledged difficulty in dating Brazilian vernacular architecture due
to its marked continuity and stability. This brings attention to the
matter of how canonical examples of Brazilian art and architecture were
cherry-picked, then oftentimes tampered with, to conform to certain
expectations regarding national character. The authors' aesthetic
movements are less relevant to how each addressed the matter of
Brazilianness in art, than is their understanding of the nature of
artistic and building professions.

Brazil in the 1880s was an empire ruled by landed elites from the
Northeast and, to a lesser degree, wealthy merchants from Rio. The only
monarchy among the republics of the American continent, she owed this
peculiar situation to the historic roots of independence. Brazil had
been a Portuguese colony from 1500 until 1808. In that year, the
Portuguese Crown, fleeing the Napoleonic invasion, relocated to its
colony, making Rio de Janeiro the seat of the Portuguese Empire. King D.
João VI strove to turn the city into a proper European capital, and thus
it came to be that a group of French artists, purged in the Restoration,
made their way to Rio, later to create the Imperial Academy of Fine Arts
on the model of its Parisian namesake. Imbued with such civilizing
mission, it is no wonder that these artists had no knowledge of, and
little interest in, Brazil's artistic and architectural past. Outside
Rio, however, art and architecture were only marginally influenced by
the academic novelties for most of the nineteenth century.

The period of our study corresponds to the transition, in Brazil, from
the old imperial gentry to rule by another rural elite in the Southeast,
and then to the hegemony of industrial capitalism. It also covers the
decline of the Imperial Fine Arts Academy, later National School of Fine
Arts. In a curious amalgam of echoes from European artistic movements,
the \emph{Salons} of Fine Arts in Rio during the late nineteenth and
early twentieth centuries were dominated by Romantic and Impressionist
painting, and Neoclassical sculpture. Urban growth and renewal displayed
eclectic styles of architecture of mostly French, British and Italian
influence, happily mixed and matched until the 1910s, when the moralist
traditional or neocolonial movement began to set the tone of
architectural debate.

\hypertarget{assessing-national-character}{%
\section{Assessing National
Character}\label{assessing-national-character}}

The Brazilian cultural establishment in the early twentieth century took
a keen interest in the matter of national character in art and
architecture. This interest can be traced back to two concurrent
influences: first, the nation-building debates spearheaded by the
Brazilian Historical and Geographic Institute (\texttt{IHGB}) in the
1840s; second, the European romanticism's nationalistic themes. A
landmark in the first factor was the publication, in 1854, of Francisco
Adolfo de Varnhagen's \emph{História Geral do Brasil} (General History
of Brazil), defining the Brazilian nation as the junction of three
races---white Portuguese, black Africans, and Amerindians. Nationalistic
romanticism had its most acclaimed expression in the literary movement
known as Indianism. The lead of novels such as José de Alencar's
(1829--1877) \emph{O Guarani} (1857), idealizing the contacts between
Portuguese colonists and the natives, was followed in music. Carlos
Gomes's 1870 opera version of that novel preceded Chiquinha Gonzaga's
erudite stylization of Afro-Brazilian songs. Even more so, painting
picked up the subject, as in Rodolfo Amoedo's (1857--1941) series of
Indianist works from the 1880s.

By the 1880s, Indianist subjects, as well as scenes of daily life, had
begun to lend local flavor to Brazilian Beaux-Arts-style painting.
Around the same time, prominent art critic Luiz Gonzaga Duque Estrada
worried, paradoxically, about the disappearance of national character in
art. In the book that was the synthesis of his early thinking, \emph{A
Arte Brasileira} (Brazilian Art, 1888), Gonzaga Duque made a pessimistic
account of Brazilian Art:

\begin{quote}
The novels, history, and poetry of this country had no influence
whatsoever in these works, which remained impervious to the dawn of
national thought. .~.~.~.
\end{quote}

\begin{quote}
One concludes, then, that this art is missing native features and
originality, primordial qualities for the founding of a School.
\end{quote}

\begin{quote}
. . . . The defining feature of our art is cosmopolitanism, and a
nation, to have a School, needs, foremost, a national art.
\autocite[p.~258--259]{gonzagaduque:1995arte}
\end{quote}

He was not particularly excited by earlier art forms, however. Even the
monumental architecture of the colonial period (1530--1808), which so
captivated later writers, was to him:

\begin{quote}
. . . . A flagrant evidence of bad taste and lack of intelligence
.~.~.~. \autocite[p.~74]{gonzagaduque:1995arte}
\end{quote}

To their credit, Gonzaga Duque conceded, colonial painters such as
Manoel da Cunha (1737--1809) had, at the very least, a sort of crafty
authenticity about them \autocite[p.~81]{gonzagaduque:1995arte} For him,
then, differently from the European romantics, national character was
not to be extracted from the achievements of previous eras. It was
something yet to be produced out of the maturing of late
nineteenth-century artists. On this regard, even a promising young
painter such as Amoedo---to whom Gonzaga Duque commissioned his
portrait---was censured for the feebleness of his subjects. National
character was thus not to be found in any specific style, but merely in
the choice of proper subject matter, chiefly Indians
\autocite[p.~185]{gonzagaduque:1995arte}

Later on, though, in 1909, Gonzaga Duque would commend Amoedo for having
outgrown Indianism, ``vanquished by the assimilating force of a superior
environment'' \autocite[p.~13]{gonzagaduque:1929contemporaneos}:
European culture, in the form of classical nudes and mythological
scenes. In his last survey of Brazilian art, the opening speech
delivered at the 1908 \emph{Salon} in Rio de Janeiro, Gonzaga Duque
summarized a triumphal picture of national art. The occasion was
momentous: 1908 marked the centennial of the transfer of the Portuguese
Crown from Lisbon to Rio, thus marking the onset of a cycle of direct
European influence on all aspects of Brazilian culture. In Gonzaga
Duque's speech, Colonial art was no longer judged by its aesthetic
value---either good or bad---, but seen as a ``historical document'' of
utmost importance \autocite[p.~247]{gonzagaduque:1929contemporaneos}.
Conversely, French influence was no longer seen as harming the
expression of national character. On the contrary, it provided the
necessary professional expertise and cultural environment in which
national character would gradually emerge.

The colonial heritage, of Portuguese and Catholic extraction, was thus
granted the status of a mythical ancestor of contemporary national
character: one to which ritual deference was owed, but one that exerted
minimal influence on present conceptions of national art. Gonzaga Duque,
speaking before the assembled professors of the Fine Arts School as well
as the President of the Republic, was cautious in his definition of this
character:

\begin{quote}
. . . . The characteristic art, truly Brazilian, shall appear from this
admirable nature, from this golden light, from this popular soul made of
the Indian's nostalgia, the animal infallibility of the African, and the
lyrical soul of the uprooted, seafaring Portuguese.
\autocite[p.~255]{gonzagaduque:1929contemporaneos}
\end{quote}

His commentaries on the \emph{Salons} of the years 1904--1907 give,
perhaps, a more vivid picture of what he took this ``truly Brazilian''
art to be. Landscape paintings, adroitly portraying the ``admirable
nature'' with its ``golden light'', were particularly favored and
commended. Scenes of daily life drew a good lot of his attention, as
much as the conventional mythological scenes and classical studies.
Indianist subjects were not altogether disparaged, but they had ceased
to be sufficient reason for a painting or sculpture to be commended.
Religious, historical and allegorical works, supposedly the acme of
academic art, were mostly shunned by the nation's most respected art
critic.

\hypertarget{tradition-and-decay}{%
\section{Tradition and Decay}\label{tradition-and-decay}}

The years leading from the restructuring of the federal debt in 1902 to
the centennial exposition of 1908 built up a momentum of conflicting
urges: cosmopolitanism and the expression of national character,
exaltation of modernity and industrialization as well as taste for the
exuberant tropical nature, national unity and showcasing states'
identities \autocite{pereira:2011exposicao} In this context, a national
character in art and architecture was an optimistic prospect rather than
something already achieved at any point in the past or present. In
keeping with the romantic nationalist drive that first established the
debate on this topic, Brazil was seen as having started off under the
yoke of Portuguese culture, slowly blended with Indian and African
influences. National character, therefore, had not been there to begin
with, then lost to degeneration, as in the myths of origins in European
cultures. On the contrary, it would gradually emerge from the maturing
interaction of racial and geographical forces in the future rather than
in a mythical past.

This placement of national character in the future was a boon for
authors and movements attempting to commandeer the idea of constructing
\emph{Brazilianness} in the first quarter of the twentieth century. It
assumed, quite contrarily to the dominant literary mood in the second
half of the previous century, that no national character had yet been
able to grow out of the melting pot of cultural influences in the
nation. Although Gonzaga Duque forecast a natural, unconscious
development of a Brazilian art school, other writers set out as a
program to create this national style.

Around the time of Gonzaga Duque's passing in Rio, a sharp critical and
literary scene was emerging in São Paulo. One of the most acclaimed and
prolific writers in this milieu, addressing the issue of national
character both explicitly and implicitly, was José Bento Monteiro Lobato
(1882--1948). His debut literary book, \emph{Urupês} (1918)---the name
of the ultimate small, remote town in São Paulo state---, was a
tongue-in-cheek portrayal of the tension between the yearning for
modernity and cosmopolitanism in provincial towns, and the backwardness
of their agrarian society. The closing chapter of the book featured an
unforgiving criticism of the literary infatuation with authentic
national characters. In the 1910s, the foremost representative of these
characters was the \emph{caipira}, archetypal small isolated farmer sunk
into endemic poverty, who had replaced the Indian as a favorite
Brazilian literary character
\autocite[p.~208--209]{monteirolobato:1944urupes}

Monteiro Lobato was, like many of his fellow bourgeois from São Paulo,
an outspoken enthusiast of material progress and industrialization. His
writing, often disparaged in the mid-twentieth century as conservative,
was actually part of his modern enthusiasm. The \emph{non sequitur}
juxtaposition of nonsensical anecdotes from several remote towns was
itself a very modern literary formula at the time, contrasting with the
conventional and uneventful way of life portrayed in the text. It
exacerbated the naturalist tendency towards depicting individual scenes,
yet departed from that style by refusing the idealized image of the
country folk as authentic representatives of the national character
\autocite[p.~303]{silva:2013modernidade}

His criticism of the inert social and material underdevelopment of small
inland towns in \emph{Cidades Mortas} (Dead Cities, 1919) was, however,
complex. The pretentious self-sufficiency of the establishment in the
aptly-named fictional city of Oblivion, for instance, alluded to a sort
of timeless decrepitude, rather than mere underdevelopment. People in
Oblivion died of boredom \autocite[p.~25]{monteirolobato:1919cidades},
not material want. The dead towns contained:

\begin{quote}
.~.~.~. Impressions of a dead youth that vegetated in the stagnation of
the dead cities. There is also some modern stuff. But both modern and
old are worth the same---nothing
\autocite[epigram]{monteirolobato:1919cidades}
\end{quote}

This sense of hopelessness, of something that would have been yet did
not bear fruit, stroke a very strong chord with Monteiro Lobato's
contemporaries, as it clearly alluded to the material and moral
decadence of once-thriving urban centers in depleted coffee-growing
regions \autocite[p.~299]{silva:2013modernidade}

\begin{quote}
There, everything was, nothing is. No verbs are conjugated in the
present tense. Everything is preterit.
\end{quote}

\begin{quote}
A group of dying cities drag on a decrepit living, spent weeping in
today's pettiness the nostalgic greatness of yore.
\autocite[p.~8]{monteirolobato:1919cidades}
\end{quote}

The former affluence of these cities was never directly spoken of in the
book, but it was vivid in the mind of its readers. Monteiro Lobato
rejected the romantic and naturalistic ideal of the glorification of a
national---or even regional---character, embodied in traditional
society. Yet, \emph{Cidades Mortas} evidenced a different sort of
nostalgia, harking back to Gonzaga Duque's early writing: the idea that
some process of national character-building had been under way, crude
but authentic, then stifled before reaching cultural maturity. This
construction of a mythical ``time outside time''
\autocite{eliade:1966aspects} of eternal decay from an acme that never
was, set the stage for the construction of a highly malleable image of
lost \emph{Brazilianness}. Crucial to this narrative of a mythical loss
of something that never quite existed in this world was the implicitly
perceived gap that retained a tenuous link to chronological continuity
\autocite[p.~293]{silva:2012cidades} This was also well known to
Monteiro Lobato's readers and did not require explicit mention in the
text. It would have been obvious, at the time, that decay in the region
of the dead cities set on around the 1860s and was felt more strongly
from the 1880s on.

Monteiro Lobato saw in material and cultural modernization the only
means of escape from this eternal decay, thus using the heyday of the
coffee-growing urban society as little more than an abstract backdrop to
his criticism of contemporary backwardness. Meanwhile, similar ideas
about a stifled development of national character, followed by a period
of decay, and the need to reassert the greatness of Brazilian identity,
were being promoted in São Paulo by a Portuguese engineer, archaeologist
and political activist, Ricardo Severo. His presence in Brazil coincided
with the development of traditionalist movements throughout the Americas
in the early twentieth century. Rooted in the European romantic
nationalism of the mid-nineteenth century, traditionalist architecture
in the Western Hemisphere took root first in the United States, where
the Mission Style provided a template for Hispanic revivals. It then
spread southward, reaching Latin America around the time of the
centennial of independence in many of its countries
\autocite[p.~12]{amaral:1994invencion}

\hypertarget{reconstructing-brazilianness}{%
\section{Reconstructing
Brazilianness}\label{reconstructing-brazilianness}}

In 1914, Severo gave a highly influential lecture at the Artistic
Culture Society in São Paulo, titled \emph{Traditional Art in Brazil:
The House and the Temple}. He shared in Gonzaga Duque's late view that
local culture had not yet developed sufficient strength to establish a
national artistic character. Unlike the art critic, however, Severo
would not wait patiently for a national school to gradually emerge out
of the cumulative efforts of individual artists: he outlined, instead,
what would give Brazilian architecture a distinctive character right
away. Severo argued that the forms and plans implanted in the Americas
by the Portuguese colonists, chiefly derived from both Roman and Moorish
sources, were to be the basis for the establishment of a
national-traditional art in Brazil \autocite[p.~249]{azevedo:1994ideas}
He followed up this first conference with further texts, putting forward
his ideas about a new traditional Brazilian architecture. In a 1917
article in one of the country's most important magazines of the time,
Severo emphasized the adaptation of Portuguese styles in Brazil as the
source of a local authenticity:

\begin{quote}
The Portuguese always gave a particular mark to the architecture he
imported, and this phenomenon, noted by the most illustrious historians
of Portuguese art, shows up in colonial Brazil as well, where the
Baroque, said to be Jesuit, took on expressions of modest simplicity,
but with a noteworthy local mark \autocite[p.~402]{severo:1917arte}
\end{quote}

He made a point to show how churches and houses in the colonial period
displayed that sort of plainness, ``being appropriated into the local
setting and in their aspect of characteristic originality'' were to
constitute ``what is or may come to be Traditional Architecture'' in
Brazil. Severo went on to describe how the vigor of this traditional
architecture persisted in Brazil up to the time of Independence (1822).
At first, he claimed, not even French influence---once decried by the
young Gonzaga Duque---was able to stifle its vitality. According to the
Portuguese engineer, it was only around the second half of the
nineteenth century, and more strongly after the proclamation of the
Republic (1889), that this traditional setting began to erode. Severo
thus proceeded to decry the arrival of fresh immigrants at that time,
``deft stuccoists come from Italy and Portugal'' who brought
eclecticism, a façadist habit of making up ``incomprehensible styles
that shocked mostly by their disconnection with the local setting and
its destiny'' \autocite[p.~415]{severo:1917arte} The solution, he
asserted, was to reclaim an authentic national tradition, consisting in
the adaptation of old Portuguese styles, as adapted and transformed by
the influence of local climate and geography.

The argument constructed by Severo thus rested on the ideal of a
national character consisting in a timeless, natural Portuguese
adaptation to their colonial \emph{genius loci}, only to be suppressed
by an unfortunate onset of eclectic influence in the second half of the
nineteenth century. The same cycle of timeless authenticity followed by
a historical gap fostering decadence would be represented, two years
later, in Monteiro Lobato's ``dead cities''. Although the fiction writer
had a much more explicitly negative view of the backwardness of the
countryside, the Portuguese engineer was also careful not to romanticize
old houses. He made clear his case for material and even aesthetic
progress, if only tempered by traditional adaptation to the site:

\begin{quote}
Traditional Architecture does not mean, then, literal reproduction of
traditional things, of archaeological fossils, of rammed earth or cob
houses, of little adobe churches, of back streets between shacks three
fathoms deep, with door and louvered window, or of the gloomy houses in
the city centers of yore, without hygiene or aesthetic appeal
\end{quote}

\begin{quote}
Traditional art is the stylization of earlier artistic forms that
constitute at some point in time the local environment, the moral
character of a people, the hallmark of its civilization; it is the
product of a rhythmic evolution of successive cycles of art and style
.~.~.~. \autocite[p.~423--424]{severo:1917arte}
\end{quote}

Severo wrote, spoke, and designed in São Paulo, a city that woke up from
its own gloomy slumber of economic inertia as late as the 1860s, when
the construction of the railway turned the city into a major economic
hub. From a point of view taken in São Paulo, little of any
architectural significance had been built in the city between the
reconstruction of the emblematic Jesuit College in 1700, and the opening
of the railways in 1867--1871 \autocite[p.~72]{lemos:1987ecletismo}
Neoclassicism in São Paulo had been an essentially rural phenomenon in
those formerly affluent coffee-growing regions, mitigated by the
practical conservatism of local builders. As Monteiro Lobato showed in
his prose, the more traditional of these regions were decadent by the
early twentieth century. In this context, the Portuguese engineer
embraced the neoclassicism of the first two thirds or so of the
nineteenth century as a continuation of the authentic
Portuguese-Brazilian tradition.

Rio de Janeiro, in contrast, was in the 1910s the nation's capital,
twice as large a city as São Paulo, and had a much more diverse
architectural heritage. It had undergone a continuous process of urban
infill and extension throughout the nineteenth century, with a
self-conscious interest in up-to-date architecture, crowned by
large-scale urban renewal in its core starting in 1902. The differences
between traditionalist discourse in São Paulo and Rio are, thus,
probably not surprising. In the capital, this traditionalist movement
was advocated, chiefly among all its proponents, by José Mariano Filho
(1881--1946), a hygienist physician and amateur architect. He was,
incidentally, responsible for coining the word ``Neocolonial'' to
designate that movement \autocite[p.~132]{kessel:2008arquitetura}
Mariano was also instrumental in rousing public support for monumental
buildings to be designed in the neocolonial style, in addition to having
sponsored, with his personal wealth, a number of architectural
competitions biased towards the same style.

In spite of the lifelong ideological constancy of his prolific writing,
José Mariano Filho was perhaps most remembered for having provided, in
1924, traveling scholarships for a number of young architects to
document what were then the largest surviving ensembles of colonial
Brazilian architecture: the former gold-mining towns of nearby Minas
Gerais State. Among these young architects was Lucio Costa (1902--1998),
future modernist agitator, theorist, and practitioner, known worldwide
for having designed Brasilia, but who was at that time an enthusiastic
supporter of the neocolonial movement. What Costa took out of this
experience will be discussed later; what interests us for now is how the
geographic scope of the effort was evidence of how differently
traditional Brazilian architecture was perceived in Rio as compared to
São Paulo.

In late-blooming São Paulo, the colonial period and much of the
nineteenth century could be lumped together into a romantic era of
traditional authenticity, albeit one endowed with a decaying quaintness
that put off its sophisticated twentieth-century writers. In Rio,
surviving buildings from the colonial period had often been added onto,
or even disfigured, by nineteenth- and twentieth-century interventions.
The relocation of the Portuguese Crown, in 1808, and even more so the
beginnings of Beaux-Arts artistic training in 1826, were the death knell
of national character for the neocolonial proponents in the nation's
capital. Since Rio had been heavily affected by these events, they had
to look elsewhere for documentation. Minas Gerais towns such as Ouro
Preto and Diamantina---now both World Heritage sites---were, conversely,
their ideal image of a traditional urban culture, supposedly frozen in
time after the decline in gold mining robbed the region of its vitality.

Thus, Mariano's ``time outside time'' could not last long beyond the end
of the eighteenth century. The mistaken assumption that most
construction had come to an end in Ouro Preto and Diamantina after the
exhaustion of the gold mines both strengthened this cut-off date
\emph{circa} 1800, and later fostered a historic preservation ideal of
removing supposedly later accretions to colonial buildings. It also
sparked a curious search for the oldest, most remote examples of
traditional architecture, expected to be the ones least contaminated by
neoclassical affectations:

\begin{quote}
The Portuguese colonist, old friend of the sun, brought to the Brazilian
land the centuries-old experience of his race, drawn out of the contact
with the oriental civilizations, and learned above all from the Moorish
experience. Thus, in confronting the Brazilian architectural problem,
the Portuguese colonist had not the slightest hesitation. .~.~.~. During
the first two centuries of national life {[}sixteenth and seventeenth
centuries{]}, Portuguese architecture was imperceptibly adjusting itself
to the Brazilian way of life. .~.~.~. The absence of classical elements,
together with the lack of a properly skilled workforce, led the people
to improvise new practices and processes, unknown in Portugal
\autocite[p.~10]{mariannofilho:1943margem}
\end{quote}

This, at a time when Rio's last remains of the early Portuguese
occupation had been eagerly obliterated in the name of urban renewal,
and before the seventeenth-century farmhouses of São Paulo would really
catch the attention of architects, was in truth inconsequential. The
eighteenth-century gold-rush architecture of Ouro Preto and Diamantina
became, effectively, the canonical examples of Brazilian colonial
architecture, and this despite the enormous differences between the
architecture of the two settlements.

\hypertarget{modernity-and-preservation}{%
\section{Modernity and Preservation}\label{modernity-and-preservation}}

Mariano's national tradition, even more so than that of Monteiro Lobato
and Ricardo Severo, hinged on the notion of ethnicity. For him, ``the
preference of man for the architecture of his homeland'' had an
emotional source, based on domestic reminiscence and unconscious
references. He therefore deplored the Portuguese immigrants and the
Brazilians who, ``instead of proceeding like the Italians, British, or
Germans, who favor the styles of their own nations, .~.~.~. seek
intently to hide or mask their own''
\autocite[p.~32]{mariannofilho:1943margem} This hiding of the national
style, in 1943, could be applied both to eclecticism and to the
characterless and ``stateless styles'' of modern architecture. Mariano
bemoaned the modern mentality, which in abolishing the principle of
decor, reduced ``the art of building to the science of making housing'',
requiring merely efficiency and economy
\autocite[p.~15]{mariannofilho:1943margem}

Art, however, did not mean mere decoration to him. Whereas Severo in São
Paulo gave in to the contemporary taste for modern plans and massing
\autocite[p.~178]{mello:2007ricardo}, and commissioned painter José
Wasth Rodrigues a comprehensive study made almost entirely of details,
Mariano steadfastly insisted, as late as 1931, that there was something
more fundamental:

\begin{quote}
I do not care for the plastic qualities of traditional Brazilian
architecture, because what I seek in it is far above these qualities.
.~.~.~. Less of an artist than a sociologist myself, I consider
architecture to be the social instrument of nationality. I do not care
for artistic virtues, the charm of lines, or the splendor of details, by
means of which the architectural styles are expressed. What I seek are
the organic qualities, the healthy virtues, the structural fundamentals,
from which stem the perfect accord of architectural feeling with the
nation's soul \autocite[p.~64]{mariannofilho:1943margem}
\end{quote}

Modern materials and technologies, however, needed not be shunned in
this endeavor to forge a new Brazilian architecture that was to remain
firmly grounded in deeper principles, respecting its ancestral ``Roman
spirit, characterized by the constant proportion of its compositional
elements, and by its rectangular geometric projection''
\autocite[p.~124]{mariannofilho:1943margem} This ideal of material
progress rooted in social conservatism echoed in the writings of the
young Lucio Costa, his former protégé, who by the 1930s had grown to be
Mariano's ideological rival. In 1929, while still a promoter of
traditional architecture, he argued against the example of the
exceptional monumental buildings of Brazilian rococo. An art made in
Brazil by individual genius with no apparent following could not form
the basis for national character, he believed. Following his former
patron, Costa held that it was, instead, in the simple architecture of
anonymous master builders that resided the functional, technical and
aesthetic homogeneity of Brazilian character
\autocite[p.~22]{puppi:1998historia} After Lucio Costa's conversion to
modernism, he authored in 1937 an article describing what he held to be
the natural development of traditional Brazilian architecture. True to
his roots, he was speaking of residential architecture built by masons
and carpenters, which remained impervious to:

\begin{quote}
. . . . The unforeseen development of bad architecture teaching---giving
future architects a whole, confused ``technical-decorative'' education,
with no link whatsoever with life, and not explaining them the
\emph{why} of each element, nor the deep reasons that conditioned, in
each period, the appearance of common features, that is, of a style
.~.~.~. \autocite[p. 39]{costa:1937documentacao1}
\end{quote}

Because Costa did not focus his narrative on learned architecture, he
was able to circumvent the problem of ``bad teaching'', and to argue for
the occurrence of an authentic traditional architecture as far forward
as 1910. He could thereby synchronize the decay precisely with the onset
of the traditional architecture movement to which he had previously
belonged, and which he now condemned. This opposition notwithstanding,
all elements of the post-romantic nationalist narrative were represented
in his text: an original period of authentic national character,
followed by another of pretentious or sophisticated decay; the
possibility of overcoming that decay by promoting a certain
architectural movement; the defense of technical modernization and
aesthetic advance while remaining anchored in that authentic national
tradition.

A few years later, though, Lucio Costa drifted from the broad
sociological picture of national character to a romantic view favoring
individual artistic intent \autocite[p.~113]{costa:2007consideracoes}
and personal genius \autocite[2]{ferraz:1948depoimento}, both embodied
in his contemporary Oscar Niemeyer. In this, he was probably influenced
by his acquaintance and driving force behind the creation of the
National Heritage Service (\texttt{SPHAN}), modernist poet Mário de
Andrade.

Although Costa put forward the thesis of a chain of authentic
architecture broken only by the neocolonial movement, his practice as
official of the \texttt{SPHAN} effectively upheld Mariano's view that
proper traditional Brazilian architecture did not reach far beyond 1800.
In theory, this view should have fostered the protection of colonial-era
monuments and urban sites, while denying protection for
nineteenth-century structures. In practice, however, matters were a lot
trickier, and actual knowledge of colonial architecture was sparse
\autocite[p.~25]{pinheiro:2012neocolonial} On the one hand, the
continuation of colonial building practices well into the nineteenth
century, and their intermingling with neoclassical influences, had been
known to Ricardo Severo and his São Paulo colleagues. On the other hand,
documentation for most sites of historic interest was virtually
nonexistent; dating often relied on conventional wisdom about local
history as well as on \emph{a priori} assumptions regarding
pre-nineteenth-century styles. Proof of this uncertainty was that
typological studies of colonial buildings, published in the
\texttt{SPHAN} journal in the 1930s and 40s, one among which penned by
Lucio Costa himself, were unable to ascribe even so little as rough date
ranges to building types.

This entailed dramatic consequences even for those buildings meant to be
preserved. A number of supposed eclectic or neocolonial accretions to
historic churches were carelessly replaced with modern recreations of
that original ``simplicity'' heralded by the Rio neocolonial architects
themselves \autocite[p.~238]{pinheiro:2012neocolonial} Certain
nineteenth-century additions to Ouro Preto houses, such as parapets,
were removed because roof overhangs were supposed to be a mainstay of
colonial architecture; forged iron railings, on the other hand, were
mistakenly attributed to the eighteenth century and thus incorporated
into a canonical image of colonial two-story houses. In São Paulo,
campaniles were ``simplified'' and entire wings in farmhouses were
removed, in an infatuation with the ideal of volumetric simplicity
promoted by Mariano and Costa, and followed with zeal by Luís Saia
\autocite[p.~61]{mayumi:2008taipa} The preservation of elements that
seemed to prefigure modern architecture was particularly favored:

\begin{quote}
Colonial constructive devices, such as buildings on stilts, trellised
louvers, and cob on wooden frames, were associated with \emph{pilotis},
\emph{brise-soleils}, and reinforced concrete. For modernist architects,
the resemblance between their own architecture and the colonial one was
not one of appearance or effect, as was the case in neocolonial
buildings, but one of structure
\autocite[p.~188]{fonseca:2005patrimonio}
\end{quote}

Conversely, whatever departed from association with these elements fell
easily in place with a picture of eclectic architecture, especially in
its popular French-inspired styles: orthostates, articulated wall
surfaces, high pitched roofs and so on.

\hypertarget{tradition-and-the-professional-architect}{%
\section{Tradition and the Professional
Architect}\label{tradition-and-the-professional-architect}}

The onset of the Modern movement in Brazilian architecture thus entailed
a power struggle between the proponents of the neocolonial movement and
their younger rivals. Both camps deployed the same narrative regarding
the development of national character in order to promote opposing views
of architectural style and space, and both derided their rival as being
so beneath them, it was ``non-architecture''. In the meantime, a single
dissonant chord struck the debate in Rio during that time. It was Adolfo
Morales de los Ríos Filho's book \emph{Grandjean de Montigny e a
evolução da arte brasileira} (Grandjean de Montigny and the Evolution of
Brazilian Art, 1941). In this work, Morales de los Ríos (1887--1973),
director of the National Fine Arts School,went back to the later Gonzaga
Duque's positive view of early-nineteenth-century French influence on
Brazilian art. The argument was similar:

\begin{quote}
Yes, it dignified Brazilian art, fighting the neglect and ignorance of a
fledgling society, .~.~.~. and contributing to the foundation of an art
school, where it would have been difficult to create it using existing
{[}local{]} resources. \autocite[p.~157]{morales:1941grandjean}
\end{quote}

Despite being part of the architectural establishment, as director of
the most important fine arts school in the nation, Morales de los Ríos
had his own axe to grind as well. The Beaux-Arts method had been under
critical fire for well over two decades, first from the traditionalist
movement, then from the modernists. The National Fine Arts School was
somewhat open, nevertheless, to the teaching of neocolonial architecture
in the 1920s, although it was considered but one of several eclectic
styles happily used and mixed by students and teachers. Lucio Costa
himself, owing to political connections, had seized the school
directorship for a few months in 1931, before being ousted by the
professors. Moreover, urban renewal in Rio, exacerbated since 1920, was
threatening the nineteenth-century French-inspired heritage just as much
as the monuments of the colonial period.

Morales de los Ríos's arguments, however, were of an obviously different
nature that those of the neocolonial-modernist groups. Unlike the
neocolonial or modernist mavericks, he was directly implicated in the
education of a class of elite artists expected to succeed in both public
and private commissions. Thus, he defended not only the historical roots
of his school, but also the diversity and adaptability of architects in
a time of rapidly changing tastes among the public, particularly so in a
moment when support for the neocolonial style in major works had all but
disappeared. Also, the Pan-American ideology of the first three decades
of the twentieth century, which had underwritten neocolonial
architecture throughout the continent, had been replaced by introverted
populism in Brazil's fascist government led by Getúlio Vargas.
Paradoxically, this introversion was deleterious for the traditional art
movement, as it made Brazilians look away from the art of the American
continent and towards the more conventional artistic centers of Europe.
Public architecture in the Vargas régime oscillated between the stripped
classicism then popular in most European countries, and modernism, which
was being half-heartedly supported by fascist Italy at the same time. As
for the fickle bourgeois of São Paulo, they moved on to favor variations
on Art Déco, Italian rationalism, and whitewashed modernism.

Both Ricardo Severo and José Mariano Filho, on the other hand, had
advocated a sort of sociological collectivism in the architectural
profession. Severo, a republican activist who at first moved to Brazil
to avoid political persecution, expected architecture and architects to
play a role in the forging of a modern---meaning nationalist---state,
conscious and proud of its ethnic origins
\autocite[p.~29]{mello:2007ricardo} He sought to balance his
archaeological interests, which led him to favor a structuralist
cohesion of sorts between a centuries-old culture and its present
developments, and his practice as an architect, where he ultimately gave
in to the public expectations of wholly modernized, eclectic plans and
picturesque massing. Nevertheless, he was successful in fostering public
taste for such traditional Brazilian elements as the seventeenth-century
\emph{alpendres} (deep and wide colonnaded porches) and generous roof
overhangs. These features went on to become favorites of Brazilian
single-family houses throughout the twentieth century.

As for Mariano, a scion of the landed elite of the Brazilian Northeast,
architecture was a dilettante passion as much as a political cause. Free
from the need to make a living out of the profession, he was thus little
interested in matters of professional cohesion and construction
industry. Conversely, with his disposable income, he was able to fund a
considerable documentation effort, as well as publicity stunts in the
form of design competitions. In addition to this, he was a regular
contributor to the Rio press throughout the 1930s. As the prestige of
neocolonial architecture for major public works eroded during that
decade, his criticism of the Modern movement increasingly resorted to
the sort of racial and political slander expected to appeal to the heads
of the fascist government: ``architectural Judaism'' and ``communist
architecture'' were expressions used in his later writings,
\autocite[p.~41]{mariannofilho:1943margem} as well as attacks on
artistic ``freemasonry.''

Both because of his early years in the neocolonial movement, and in
reaction to Mariano's criticism of modern architecture, Lucio Costa,
too, resorted to an ethnic narrative regarding the roots of national
artistic character. As a pure-bred white Brazilian of colonial
Portuguese descent, the son of a Navy officer, he was in as strong a
position as Mariano to claim authority to speak for national roots.
Moreover, his political connections in the Vargas government freed him
from the concern with day-to-day professional practice in a market
environment. Costa at first supported Mariano's narrative of a
collective, anonymous architecture without architects, even through his
first decade as a leader of the Modern movement in Brazil. This led him
to shun at first the few known masters of Brazilian art in the colonial
period. By 1945, however, his writings focused chiefly on self-conscious
artistic intent and the importance of individual genius for the
development of style. A hinge moment in his views probably occurred
around 1939, when he supported Oscar Niemeyer's attempt to insert a
modernist hotel at the heart of the historic district in Ouro Preto,
although influence from poets Mário de Andrade and Carlos Drummond de
Andrade in the \texttt{SPHAN} is not to be excluded. Costa then moved
away from the ethnological understanding of architectural coherence, to
argue that an architectural work of art ``shall not resent the proximity
to other works of art'' \autocite[quoted in][]{comas:2010passado}
Throughout the remainder of his long writing career, he strove to
reconcile both views as the discourses on the artistic originality of
the Modern movement became hegemonic. The unchallenged ethos of national
genius that Costa helped construct for Niemeyer remains to this day a
favorite topic of debate regarding the nature of professional practice
in Brazilian architecture.

\hypertarget{conclusion}{%
\section{Conclusion}\label{conclusion}}

Despite their differences, Monteiro Lobato, Ricardo Severo, José Mariano
Filho, and Lucio Costa constructed and upheld a long-lived teleological
history of Brazilian art. It was fueled by a nationalist spirit, placed
against perceived weaknesses in Brazilian culture. The starting point of
this narrative was invariably a timeless period of formation of the
national identity. This needed not be an exemplary or admirable stage;
Monteiro Lobato despised the backwardness of the remote countryside as
well as the pretentious manners of regions that had experienced
ephemeral wealth. The essential was that it provided fundamentals of
national identity that could be later reworked and improved: Portuguese
language and way of life, adaptation to climate and geography,
simplicity, and rationality.

Then was supposed to follow a clearly circumscribed period of decay: for
Monteiro Lobato, the decline of the oldest coffee-growing regions, from
the 1860s on, was the preferred reference; Severo identified it in the
cosmopolitan burst of growth in São Paulo starting in the 1870s, fueled
by immigrants who threatened the cohesion of the old
Portuguese-Brazilian culture; Mariano, in this specific topic the most
enduringly influential among these four writers, pointed to the cultural
disruption caused by the relocation of the Portuguese Crown to Rio in
1808. Lucio Costa made a timid attempt to find fault only in the
traditional art movements of the 1910s and 20s, then retreated to the
less controversial position of Mariano's narrative.

This shunning of nineteenth-century art, or at least that of the last
third of that century, had strong consequences for Brazilian art
historiography. The colonial period had been little known up to the
documentation efforts of the traditionalists, but vernacular
architecture and art of the nineteenth century remained poorly studied
throughout most of the twentieth century. While documentation for the
high art of the same period fortunately survived, several important
buildings were allowed to be destroyed, because they did not fit into
the continuous march of national character through history. Lucio Costa
himself, shortly before retiring from his Heritage office, wrote an
explicit refusal to list the former Senate building in Rio, demolished
to make way for a subway station. Research on nineteenth-century art and
architecture has flourished in Brazil over the past two decades, and the
writings of Monteiro Lobato, Severo, Mariano and Costa have been
reappraised as important historical documents. Incidentally, this has
led to an unfortunate reaction portraying the period between 1930 and
1990 as a dark valley in Brazilian art historiography
\autocite{puppi:1998historia} Meanwhile, the contributions of Gonzaga
Duque and Morales de los Ríos to the study of Brazilian art have yet to
receive major scholarly attention. The Rio art critic's move away from
teleological theses and towards circumstantial criticism over his career
does not lend itself to the far-reaching historical revisions that have
been popular in recent years. As for the Beaux-Arts architect and
teacher, despite having occupied high-profile offices during his career,
he was eclipsed by the modern architects. When the hegemony of the
Modern movement faded away, he was then placed in the shadow of his
father, the Spanish architect who designed some of Rio's finest eclectic
buildings.
