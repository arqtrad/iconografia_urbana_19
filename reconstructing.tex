\hypertarget{reconstructing-brazilianness}{%
\section{Reconstructing
Brazilianness}\label{reconstructing-brazilianness}}

In 1914, Severo gave a highly influential lecture at the Artistic
Culture Society in São Paulo, titled \emph{Traditional Art in Brazil:
The House and the Temple}. He shared in Gonzaga Duque's late view that
local culture had not yet developed sufficient strength to establish a
national artistic character. Unlike the art critic, however, Severo
would not wait patiently for a national school to gradually emerge out
of the cumulative efforts of individual artists: he outlined, instead,
what would give Brazilian architecture a distinctive character right
away. Severo argued that the forms and plans implanted in the Americas
by the Portuguese colonists, chiefly derived from both Roman and Moorish
sources, were to be the basis for the establishment of a
national-traditional art in Brazil \autocite[p.~249]{azevedo:1994ideas}
He followed up this first conference with further texts, putting forward
his ideas about a new traditional Brazilian architecture. In a 1917
article in one of the country's most important magazines of the time,
Severo emphasized the adaptation of Portuguese styles in Brazil as the
source of a local authenticity:

\begin{quote}
The Portuguese always gave a particular mark to the architecture he
imported, and this phenomenon, noted by the most illustrious historians
of Portuguese art, shows up in colonial Brazil as well, where the
Baroque, said to be Jesuit, took on expressions of modest simplicity,
but with a noteworthy local mark \autocite[p. 402]{severo:1917arte}
\end{quote}

He made a point to show how churches and houses in the colonial period
displayed that sort of plainness, ``being appropriated into the local
setting and in their aspect of characteristic originality'' were to
constitute ``what is or may come to be Traditional Architecture'' in
Brazil. Severo went on to describe how the vigor of this traditional
architecture persisted in Brazil up to the time of Independence (1822).
At first, he claimed, not even French influence---once decried by the
young Gonzaga Duque---was able to stifle its vitality. According to the
Portuguese engineer, it was only around the second half of the
nineteenth century, and more strongly after the proclamation of the
Republic (1889), that this traditional setting began to erode. Severo
thus proceeded to decry the arrival of fresh immigrants at that time,
``deft stuccoists come from Italy and Portugal'' who brought
eclecticism, a façadist habit of making up ``incomprehensible styles
that shocked mostly by their disconnection with the local setting and
its destiny'' \autocite[p.~415]{severo:1917arte} The solution, he
asserted, was to reclaim an authentic national tradition, consisting in
the adaptation of old Portuguese styles, as adapted and transformed by
the influence of local climate and geography.

The argument constructed by Severo thus rested on the ideal of a
national character consisting in a timeless, natural Portuguese
adaptation to their colonial \emph{genius loci}, only to be suppressed
by an unfortunate onset of eclectic influence in the second half of the
nineteenth century. The same cycle of timeless authenticity followed by
a historical gap fostering decadence would be represented, two years
later, in Monteiro Lobato's ``dead cities''. Although the fiction writer
had a much more explicitly negative view of the backwardness of the
countryside, the Portuguese engineer was also careful not to romanticize
old houses. He made clear his case for material and even aesthetic
progress, if only tempered by traditional adaptation to the site:

\begin{quote}
Traditional Architecture does not mean, then, literal reproduction of
traditional things, of archaeological fossils, of rammed earth or cob
houses, of little adobe churches, of back streets between shacks three
fathoms deep, with door and louvered window, or of the gloomy houses in
the city centers of yore, without hygiene or aesthetic appeal

Traditional art is the stylization of earlier artistic forms that
constitute at some point in time the local environment, the moral
character of a people, the hallmark of its civilization; it is the
product of a rhythmic evolution of successive cycles of art and style
.~.~.~. \autocite[p.~423--424]{severo:1917arte}
\end{quote}

Severo wrote, spoke, and designed in São Paulo, a city that woke up from
its own gloomy slumber of economic inertia as late as the 1860s, when
the construction of the railway turned the city into a major economic
hub. From a point of view taken in São Paulo, little of any
architectural significance had been built in the city between the
reconstruction of the emblematic Jesuit College in 1700, and the opening
of the railways in 1867--1871 \autocite[p.~72]{lemos:1987ecletismo}
Neoclassicism in São Paulo had been an essentially rural phenomenon in
those formerly affluent coffee-growing regions, mitigated by the
practical conservatism of local builders. As Monteiro Lobato showed in
his prose, the more traditional of these regions were decadent by the
early twentieth century. In this context, the Portuguese engineer
embraced the neoclassicism of the first two thirds or so of the
nineteenth century as a continuation of the authentic
Portuguese-Brazilian tradition.

Rio de Janeiro, in contrast, was in the 1910s the nation's capital,
twice as large a city as São Paulo, and had a much more diverse
architectural heritage. It had undergone a continuous process of urban
infill and extension throughout the nineteenth century, with a
self-conscious interest in up-to-date architecture, crowned by
large-scale urban renewal in its core starting in 1902. The differences
between traditionalist discourse in São Paulo and Rio are, thus,
probably not surprising. In the capital, this traditionalist movement
was advocated, chiefly among all its proponents, by José Mariano Filho
(1881--1946), a hygienist physician and amateur architect. He was,
incidentally, responsible for coining the word ``Neocolonial'' to
designate that movement \autocite[p.~132]{kessel:2008arquitetura}
Mariano was also instrumental in rousing public support for monumental
buildings to be designed in the neocolonial style, in addition to having
sponsored, with his personal wealth, a number of architectural
competitions biased towards the same style.

In spite of the lifelong ideological constancy of his prolific writing,
José Mariano Filho was perhaps most remembered for having provided, in
1924, traveling scholarships for a number of young architects to
document what were then the largest surviving ensembles of colonial
Brazilian architecture: the former gold-mining towns of nearby Minas
Gerais State. Among these young architects was Lucio Costa (1902--1998),
future modernist agitator, theorist, and practitioner, known worldwide
for having designed Brasilia, but who was at that time an enthusiastic
supporter of the neocolonial movement. What Costa took out of this
experience will be discussed later; what interests us for now is how the
geographic scope of the effort was evidence of how differently
traditional Brazilian architecture was perceived in Rio as compared to
São Paulo.
