\hypertarget{resumo}{%
\section{Resumo}\label{resumo}}

A representação de vistas urbanas na pintura brasileira do século XIX
acompanha os debates formativos de uma ideologia da nacionalidade,
articulando-se, também, com discursos do poder político e com o
conceito, contraditório a essa expressão de poder, da natureza sublime.
As cenas da cidade brasileira pintadas ao longo do século se articulam,
assim, em três momentos: um primeiro, de inserção nas demandas de arte
cívica por parte do poder político, um segundo de afirmação do
romantismo artístico tendo a natureza como elemento dominante, e um
terceiro de síntese e ressignificação dos conceitos anteriores.

\textbf{Palavras-chave:} Iconografia, cidade brasileira, século XIX,
Romantismo, Realismo

\hypertarget{abstract}{%
\section{Abstract}\label{abstract}}

The representation of cityscapes in Brazilian painting during the
nineteenth century follows the nation-building debates of that time,
also addressing the discourses of political power and the concept,
contradicting the first, of sublime nature. The Brazilian cityscapes
produced all along that century define three distinct periods: first,
the insertion of painting within the demands of civic art by the power
of the state; second, the affirmation of artistic Romanticism holding up
nature as the dominant element; finally, a cycle of synthesis and
reformulation of the meanings previously constructed.

\textbf{Keywords:} Iconography, Brazilian city, Nineteenth century,
Romanticism, Realism
