\hypertarget{arte-nacional-como-problema-de-estilo}{%
\section{1910--1940: Arte nacional como problema de
estilo}\label{arte-nacional-como-problema-de-estilo}}

\hypertarget{reconstruindo-a-brasilidade}{%
\subsection{Reconstruindo a
Brasilidade}\label{reconstruindo-a-brasilidade}}

Em contraste com a São Paulo de Monteiro Lobato, a qual ainda lutava
para se desvencilhar do atraso criticado pelo autor, Rio de Janeiro na
década de 1910 era a cosmopolita capital da República, com o dobro da
população da capital paulista. A riqueza da herança artística e cultural
carioca constituía um contexto bem diverso diante do qual os autores
nacionalistas da capital federal se posicionavam. O mais profícuo
representante de um discurso sobre o caráter nacional em arquitetura no
Rio era o médico sanitarista e arquiteto amador José Mariano Filho
(1881--1946), irmão do poeta Olegário Mariano. Criador do termo
``neocolonial'' \autocite[p.~132]{kessel:2008arquitetura}, Mariano foi o
grande divulgador do movimento homônimo, tendo patrocinado, com sua
fortuna pessoal, concursos e publicações.

Se em São Paulo a segunda metade do século XIX era o marco do declínio
de uma cultura brasileira em gestação, no Rio a pujança cultural do
período excluía a mesma interpretação. Na recém industrializada São
Paulo, o período colonial e boa parte do século XIX podiam, de fato, ser
amalgamados numa era romântica de autenticidade tradicional, ainda que
envolta numa névoa de decadência e desconhecimento. No Rio de Janeiro, a
instalação da Corte portuguesa, e mais ainda a abertura da Academia
Imperial de Belas-Artes em 1826, passaram a ser vistos pelo movimento
tradicionalista como as causas da extinção de uma arte verdadeiramente
nacional. As cidades do ouro em Minas Gerais, por outro lado, apareciam
como um retrato ideal desse ``tempo fora do tempo'', supostamente
congelado com o declínio da mineração ao final do século XVIII. O
próprio Mariano patrocinou essa corrida aos mais remotos vestígios ---
geográfica e cronologicamente --- de arquitetura colonial, que se
supunha serem os menos contaminados pelos galicismos do século XIX:

\begin{quote}
O colonisador português, velho amigo do sol, trouxe para a terra
brasileira a experiência secular da raça, haurida do contato com as
civilisações orientais, e instruida sobretudo, na experiência mourisca.
{[}\ldots{]} Assim, durante os dois primeiros séculos de vida nacional
insensivelmente se foi reajustando a arquitetura lusa ao viver
brasileiro. {[}\ldots{]}

A ausência de elementos clássicos, de par com a carência de mão de obra
adequada leva o povo a improvisar novas praxes e processos desconhecidos
em Portugal. \autocite[p.~9--10]{mariannofilho:1943margem}
\end{quote}

A noção de tradição nacional na obra de Mariano, portanto, girava em
torno da questão racial. Ele lamentava os imigrantes portugueses e os
luso-brasileiros que, ``ao invés de proceder como os italianos,
ingleses, ou alemães, que preferem os estilos da própria nacionalidade,
{[}\ldots{]} procuram insistentemente disfarçá-lo, ou mascará-lo''
\autocite[p.~32]{mariannofilho:1943margem}. Por outro lado, a expressão
do caráter nacional era, para ele, independente dos detalhes decorativos
ou mesmo dos materiais históricos, num discurso surpreendentemente
alinhado com os ideais modernistas da mesma época. Em texto de 1931,
Mariano insistia que:

\begin{quote}
Eu não me preocupo jamais com as qualidades plásticas da arquitetura
tradicional brasileira, porque o que eu busco nela, está muito acima
dessas qualidades. {[}\ldots{]} Menos artista que sociólogo, eu
considero a arquitetura como instrumento social da nacionalidade. Não me
preocupo com as virtudes artísticas, com o encanto das linhas, ou o
esplendor dos detalhes, por meio dos quais se expressam os estilos
arquitetônicos. O que eu busco, são as qualidades orgânicas, as virtudes
sadias, os fundamentos estruturais, dos quais resultam a perfeita
concordância do sentimento arquitetônico com a alma da nação.
\autocite[p.~64]{mariannofilho:1943margem}
\end{quote}

Esse ideal de progresso material vinculado a um conservadorismo social
se refletiu nos escritos de um discípulo de Mariano, o arquiteto Lucio
Costa (1902--1998). Em 1929, provavelmente ainda engajado no movimento
neocolonial, Costa afirmara que as obras monumentais e individualistas
do Aleijadinho não estavam imbuídas de um verdadeiro caráter nacional.
Seria, ao contrário, na arquitetura simples dos mestres de obras
anônimos que residia a homogeneidade técnica, funcional e estética do
caráter brasileiro \autocite[p.~22]{puppi:1998historia}. Após a sua
``conversão'' ao modernismo, Costa escreveu outro artigo, em 1937,
descrevendo sua ideia de um desenvolvimento histórico da arquitetura
brasileira. Ainda fiel às suas raízes, ele tratava exclusivamente da
arquitetura residencial dos mestres de obras, que permaneceram imunes
ao:

\begin{quote}
{[}\ldots{]} imprevisto desenvolvimento do mau ensino da arquitetura ---
dando-se aos futuros arquitetos tôda uma confusa bagagem
``técnico-decorativa'', sem qualquer ligação com a vida, e não se lhes
explicando direito o \emph{porquê} de cada elemento, nem as razões
profundas que condicionaram, em cada época, o aparecimento de
características comuns, ou seja, de um estilo {[}\ldots{]}
\autocite[p.~39]{costa:1937documentacao1}
\end{quote}

Como Lucio Costa não argumentava em torno da arquitetura erudita, ele
pôde contornar o problema do ``mau ensino'', admitindo a sobrevivência
de uma arquitetura tradicional autêntica até 1910. Isso lhe permitiu,
então, sincronizar o declínio do autêntico caráter nacional precisamente
com o início dos movimentos tradicionalistas que ele agora renegava. Não
obstante essa oposição ideológica, todos os elementos da narrativa
nacionalista pós-romântica, já vistos nos textos de Gonzaga Duque,
Monteiro Lobato e José Mariano Filho, estavam presentes na obra de
Costa: a existência de um período de autêntico caráter nacional ---
juízos de valor estético à parte ---, seguido por outro de decrepitude
pretensiosa ou sofisticada; a possibilidade de se superar essa
decrepitude promovendo um determinado movimento arquitetônico; e a
defesa da modernização técnica e de um progresso estético de algum modo
vinculados, por meio de referências literárias bastante abstratas e
imprecisas, àquela autêntica tradição nacional.
