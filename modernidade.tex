\hypertarget{modernity-and-preservation}{%
\section{Modernity and Preservation}\label{modernity-and-preservation}}

Mariano's national tradition, even more so than that of Monteiro Lobato
and Ricardo Severo, hinged on the notion of ethnicity. For him, ``the
preference of man for the architecture of his homeland'' had an
emotional source, based on domestic reminiscence and unconscious
references. He therefore deplored the Portuguese immigrants and the
Brazilians who, ``instead of proceeding like the Italians, British, or
Germans, who favor the styles of their own nations, .~.~.~. seek
intently to hide or mask their own''
\autocite[p.~32]{mariannofilho:1943margem} This hiding of the national
style, in 1943, could be applied both to eclecticism and to the
characterless and ``stateless styles'' of modern architecture. Mariano
bemoaned the modern mentality, which in abolishing the principle of
decor, reduced ``the art of building to the science of making housing'',
requiring merely efficiency and economy
\autocite[p.~15]{mariannofilho:1943margem}

Art, however, did not mean mere decoration to him. Whereas Severo in São
Paulo gave in to the contemporary taste for modern plans and massing
\autocite[p.~178]{mello:2007ricardo}, and commissioned painter José
Wasth Rodrigues a comprehensive study made almost entirely of details,
Mariano steadfastly insisted, as late as 1931, that there was something
more fundamental:

\begin{quote}
I do not care for the plastic qualities of traditional Brazilian
architecture, because what I seek in it is far above these qualities.
.~.~.~. Less of an artist than a sociologist myself, I consider
architecture to be the social instrument of nationality. I do not care
for artistic virtues, the charm of lines, or the splendor of details, by
means of which the architectural styles are expressed. What I seek are
the organic qualities, the healthy virtues, the structural fundamentals,
from which stem the perfect accord of architectural feeling with the
nation's soul \autocite[p.~64]{mariannofilho:1943margem}
\end{quote}

Modern materials and technologies, however, needed not be shunned in
this endeavor to forge a new Brazilian architecture that was to remain
firmly grounded in deeper principles, respecting its ancestral ``Roman
spirit, characterized by the constant proportion of its compositional
elements, and by its rectangular geometric projection''
\autocite[p.~124]{mariannofilho:1943margem} This ideal of material
progress rooted in social conservatism echoed in the writings of the
young Lucio Costa, his former protégé, who by the 1930s had grown to be
Mariano's ideological rival. In 1929, while still a promoter of
traditional architecture, he argued against the example of the
exceptional monumental buildings of Brazilian rococo. An art made in
Brazil by individual genius with no apparent following could not form
the basis for national character, he believed. Following his former
patron, Costa held that it was, instead, in the simple architecture of
anonymous master builders that resided the functional, technical and
aesthetic homogeneity of Brazilian character
\autocite[p.~22]{puppi:1998historia} After Lucio Costa's conversion to
modernism, he authored in 1937 an article describing what he held to be
the natural development of traditional Brazilian architecture. True to
his roots, he was speaking of residential architecture built by masons
and carpenters, which remained impervious to:

\begin{quote}
. . . . The unforeseen development of bad architecture teaching---giving
future architects a whole, confused ``technical-decorative'' education,
with no link whatsoever with life, and not explaining them the
\emph{why} of each element, nor the deep reasons that conditioned, in
each period, the appearance of common features, that is, of a style
.~.~.~. \autocite[p.~39]{costa:1937documentacao1}
\end{quote}

Because Costa did not focus his narrative on learned architecture, he
was able to circumvent the problem of ``bad teaching'', and to argue for
the occurrence of an authentic traditional architecture as far forward
as 1910. He could thereby synchronize the decay precisely with the onset
of the traditional architecture movement to which he had previously
belonged, and which he now condemned. This opposition notwithstanding,
all elements of the post-romantic nationalist narrative were represented
in his text: an original period of authentic national character,
followed by another of pretentious or sophisticated decay; the
possibility of overcoming that decay by promoting a certain
architectural movement; the defense of technical modernization and
aesthetic advance while remaining anchored in that authentic national
tradition.

A few years later, though, Lucio Costa drifted from the broad
sociological picture of national character to a romantic view favoring
individual artistic intent \autocite[p.~113]{costa:2007consideracoes}
and personal genius \autocite[2]{ferraz:1948depoimento}, both embodied
in his contemporary Oscar Niemeyer. In this, he was probably influenced
by his acquaintance and driving force behind the creation of the
National Heritage Service (\texttt{SPHAN}), modernist poet Mário de
Andrade.

Although Costa put forward the thesis of a chain of authentic
architecture broken only by the neocolonial movement, his practice as
official of the \texttt{SPHAN} effectively upheld Mariano's view that
proper traditional Brazilian architecture did not reach far beyond 1800.
In theory, this view should have fostered the protection of colonial-era
monuments and urban sites, while denying protection for
nineteenth-century structures. In practice, however, matters were a lot
trickier, and actual knowledge of colonial architecture was sparse
\autocite[p.~25]{pinheiro:2012neocolonial} On the one hand, the
continuation of colonial building practices well into the nineteenth
century, and their intermingling with neoclassical influences, had been
known to Ricardo Severo and his São Paulo colleagues. On the other hand,
documentation for most sites of historic interest was virtually
nonexistent; dating often relied on conventional wisdom about local
history as well as on \emph{a priori} assumptions regarding
pre-nineteenth-century styles. Proof of this uncertainty was that
typological studies of colonial buildings, published in the
\texttt{SPHAN} journal in the 1930s and 40s, one among which penned by
Lucio Costa himself, were unable to ascribe even so little as rough date
ranges to building types.

This entailed dramatic consequences even for those buildings meant to be
preserved. A number of supposed eclectic or neocolonial accretions to
historic churches were carelessly replaced with modern recreations of
that original ``simplicity'' heralded by the Rio neocolonial architects
themselves \autocite[p.~238]{pinheiro:2012neocolonial} Certain
nineteenth-century additions to Ouro Preto houses, such as parapets,
were removed because roof overhangs were supposed to be a mainstay of
colonial architecture; forged iron railings, on the other hand, were
mistakenly attributed to the eighteenth century and thus incorporated
into a canonical image of colonial two-story houses. In São Paulo,
campaniles were ``simplified'' and entire wings in farmhouses were
removed, in an infatuation with the ideal of volumetric simplicity
promoted by Mariano and Costa, and followed with zeal by Luís Saia
\autocite[p.~61]{mayumi:2008taipa} The preservation of elements that
seemed to prefigure modern architecture was particularly favored:

\begin{quote}
Colonial constructive devices, such as buildings on stilts, trellised
louvers, and cob on wooden frames, were associated with \emph{pilotis},
\emph{brise-soleils}, and reinforced concrete. For modernist architects,
the resemblance between their own architecture and the colonial one was
not one of appearance or effect, as was the case in neocolonial
buildings, but one of structure
\autocite[p.~188]{fonseca:2005patrimonio}
\end{quote}

Conversely, whatever departed from association with these elements fell
easily in place with a picture of eclectic architecture, especially in
its popular French-inspired styles: orthostates, articulated wall
surfaces, high pitched roofs and so on.
