\hypertarget{considerauxe7uxf5es-finais}{%
\section{Considerações Finais}\label{considerauxe7uxf5es-finais}}

Apesar de suas diferentes vinculações ideológicas, profissionais e
estilísticas, Monteiro Lobato, José Mariano Filho e Lucio Costa
construíram uma duradoura teleologia histórica da arte brasileira. Essa
visão era alimentada por um espírito nacionalista tencionando superar
uma noção de que a cultura brasileira era fraca ou incipiente. O ponto
de partida, sugerido nos escritos de juventude de Gonzaga Duque, era
invariavelmente a existência de um período atemporal de gestação da
identidade nacional --- sem que essa gestação tivesse logrado dar à luz
uma imagem acabada e exemplar de brasilidade. Assim, Monteiro Lobato
desprezava o longínquo subdesenvolvimento do interior que expressaria
esse caráter. Seguia-se, no discurso dos três, um período de declínio
claramente circunscrito: fosse a retração da riqueza cafeeira para
Monteiro Lobato, os estrangeirismos fomentados pela Academia Imperial de
Belas Artes na visão de Mariano, ou o próprio movimento neocolonial no
texto de Lucio Costa.

\begin{center}\rule{0.5\linewidth}{0.5pt}\end{center}

Then was supposed to follow a clearly circumscribed period of decay: for
Monteiro Lobato, the decline of the oldest coffee-growing regions, from
the 1860s on, was the preferred reference; Severo identified it in the
cosmopolitan burst of growth in São Paulo starting in the 1870s, fueled
by immigrants who threatened the cohesion of the old
Portuguese-Brazilian culture; Mariano, in this specific topic the most
enduringly influential among these four writers, pointed to the cultural
disruption caused by the relocation of the Portuguese Crown to Rio in
1808. Lucio Costa made a timid attempt to find fault only in the
traditional art movements of the 1910s and 20s, then retreated to the
less controversial position of Mariano's narrative.

This shunning of nineteenth-century art, or at least that of the last
third of that century, had strong consequences for Brazilian art
historiography. The colonial period had been little known up to the
documentation efforts of the traditionalists, but vernacular
architecture and art of the nineteenth century remained poorly studied
throughout most of the twentieth century. While documentation for the
high art of the same period fortunately survived, several important
buildings were allowed to be destroyed, because they did not fit into
the continuous march of national character through history. Lucio Costa
himself, shortly before retiring from his Heritage office, wrote an
explicit refusal to list the former Senate building in Rio, demolished
to make way for a subway station. Research on nineteenth-century art and
architecture has flourished in Brazil over the past two decades, and the
writings of Monteiro Lobato, Severo, Mariano and Costa have been
reappraised as important historical documents.

\begin{center}\rule{0.5\linewidth}{0.5pt}\end{center}

Esse desprezo pela produção artística da segunda metade do século XIX e
do início do século XX, teve consequências tenebrosas para a
historiografia da arte brasileira durante várias décadas.
Reciprocamente, a valorização desse período antes rejeitado, na
historiografia das duas ou três últimas décadas, teve um infeliz
corolário. Foi a historiografia da arte e da arquitetura no Brasil entre
1930 e 1990 que passou a ser rejeitada em bloco por historiadores
revisionistas como Marcelo Puppi \autocite{puppi:1998historia}. Ainda
assim, a contribuição de Gonzaga Duque para o estudo da arte brasileira
ainda não recebeu a devida atenção acadêmica --- e o crítico de arte foi
relegado ao mesmo papel subalterno de ``documento histórico'' do início
do século XX ao qual ele relegara a arte do período colonial.

\begin{center}\rule{0.5\linewidth}{0.5pt}\end{center}

As for the Beaux-Arts architect and teacher, despite having occupied
high-profile offices during his career, he was eclipsed by the modern
architects. When the hegemony of the Modern movement faded away, he was
then placed in the shadow of his father, the Spanish architect who
designed some of Rio's finest eclectic buildings.

\begin{center}\rule{0.5\linewidth}{0.5pt}\end{center}

A cidade é apresentada sob o prisma de sua unidade num panorama distante
ou na representatividade de um monumento, ao passo que os conflitos
sociais e a identidade dos lugares --- capturados pelos viajantes
estrangeiros do início do século XIX --- desaparecem temporariamente do
registro visual.

Trata-se, ao mesmo tempo, de uma visão que aplasta as diferenças
políticas, sociais e regionais. Será nos primórdios da crítica de arte
no Brasil, e com a influência do impressionismo e do realismo na
pintura, que se perde essa dupla leitura do projeto civilizador e de sua
limitação diante do sublime. Nesse momento, ao final do século XIX, as
perspectivas se restringem, a fotografia documenta os estertores de um
regime político, da instituição escravista e de bairros inteiros
apagados pela marcha do progresso republicano.
