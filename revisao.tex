A temática nacional no discurso artístico brasileiro do século XIX se
assenta na historicização da crítica artística e literária que se opera
imediatamente após a independência do Brasil. Esta perspectiva é posta
em prática desde o \emph{estado da arte} apresentado por Gonçalves de
Magalhães, Francisco Torres Homem de Mello e Araújo Porto-Alegre em 1834
\autocite{magalhaes:1834resume}. Ao fazer frente ao ``projeto
histórico-cultural europeu enquanto paradigma universal'' que ``condena
a cultura latinoamericana a ser lida como repetição {[}\ldots{]}
imperfeita'' \autocite[p.~360--361]{dallemand:1996urban15}, a crítica de
arte e de arquitetura no Império e na primeira República reivindica uma
consciência da trajetória histórica brasileira como chancela da sua
independência cultural.

Ultrapassando o escopo restritivo da encomenda oficial, a temática da
cidade na literatura e na pintura passa a representar, no âmbito da
encomenda privada ou da produção artística autônoma, a imagem dos
olhares cambiantes lançados pelas elites intelectuais e econômicas sobre
a cidade. Tais olhares, ainda que pautados por uma ideologia do
progresso material burguês, não coincidem integralmente com os já
bastante estudados discursos técnicos sobre ``melhoramentos urbanos''
\autocite{salgueiro:2001cidades} mas abarcam, também, senão
prioritariamente, a formulação ideológica do caráter nacional em termos
análogos aos da historiografia então promovida pelo Instituo Histórico e
Geográfico Brasileiro (IHGB).

A busca pelo modo ``Como se deve escrever a história do Brazil'',
lançada pelo IHGB e respondida por von Martius, tem a fé no progressivo
dealbar da identidade nacional como sentido explicativo da história e,
simultaneamente, como tema a ser tratado na historiografia
\autocite[p.~402]{martius:1845se6}. O discurso artístico brasileiro no
final do Império e na primeira República apresenta, em contraste, uma
marcada insegurança quanto à consecução de um caráter artístico
verdadeiramente nacional e ao seu valor estético. Prenunciando a
``retórica da perda'' que formará a justificativa fundadora da
preservação patrimonial no Estado Novo e além
\autocite[p.~90]{goncalves:1996retorica}, autores e artistas de finais
do século XIX e início do XX, indistintamente tradicionalistas e
modernistas, denunciam a falta de um caráter devidamente nacional na
arte e na arquitetura brasileiras do século XIX.

O discurso retrospectivo sobre a cidade e a arquitetura brasileiras
comparece de modo mais explícito, em meados do século XIX, por meio do
gênero literário e documental da biografia. Aquela do Mestre Valentim,
por Manuel de Araújo Porto-Alegre (1856), e a do Aleijadinho, por
Rodrigo José Ferreira Bretas (1858), são marcos célebres da
historiografia da arte brasileira. Ao contrário dos estudos etnográficos
como o da Comissão científica de exploração liderada por Gonçalves Dias
\autocite{kury:2001comissao}, todavia, a produção pictórica do século
XIX recebe atenção mais jornalística que historiográfica.

A ausência de ``brasilidade'' na arte brasileira e a carência de valores
artísticos --- no sentido normativo academicista --- na cultura nacional
revezam-se como questões subjacentes à crítica de arte desde a sua
consagração como gênero literário, em torno à publicação d'\emph{A arte
brasileira} de Gonzaga Duque em 1888, até o derradeiro elogio da
tradição acadêmica em \emph{Grandjean de Montigny e a evolução da arte
brasileira} por Adolfo Morales de los Ríos Filho, publicado em 1941.
Neste recorte cronológico, tem lugar a instituição de um cânone da
``arte brasileira'' a partir da síntese dos escassos conhecimentos
históricos e arqueológicos então disponíveis. Este cânone, adstrito à
Bahia e ao Rio de Janeiro vice-reais, ao território paulista do
``segundo século'' e, por fim, aos grandes centros auríferos do sul de
Minas Gerais, não perdeu substancialmente sua hegemonia historiográfica
apesar da vasta ampliação do horizonte documental ao longo do século XX.

À uniformidade do cânone colonial corresponde um consenso crítico acerca
da arte do século XIX, identificando esse século como uma
\emph{travessia do deserto} no processo formativo do caráter artístico
nacional. Neste ponto, todavia, emergem discordâncias quanto à
periodização. Gonzaga Duque em 1888 \autocite*{gonzagaduque:1995arte}
recorta um ``breve século XIX'' que vai da introdução do academicismo
sob a Missão artística francesa (1816) à eclosão do naturalismo
literário e artístico, associado à obra de Aluísio Azevedo e Belmiro de
Almeida, seus contemporâneos. Adolfo Morales de los Ríos Filho
\autocite*{morales:1941grandjean} corrobora esta cronologia por motivos
mais estritamente disciplinares da profissão arquitetônica. Lucio Costa
\autocite*{costa:2007aleijadinho} contrapõe-lhe em 1929 um ``longo
século XIX'' articulado entre o declínio na produção arquitetônica
mineira, sinalizada pelo esplendor tardio do Aleijadinho (ativo a partir
de c.~1765), e a afirmação da arquitetura neocolonial que ele próprio
protagoniza a partir de 1924. Mais tarde, porém, Costa adere à leitura
de Ricardo Severo \autocite*{severo:1917arte}, o qual situa a ruptura da
tradição artística no ecletismo trazido pelos imigrantes a partir da
década de 1870. Entre esses extremos, Monteiro Lobato, em \emph{Cidades
mortas} \autocite*{monteirolobato:1919cidades}, observa o lento declínio
econômico e cultural do vale do Paraíba ao longo da segunda metade do
século XIX. Esses cinco autores constituem as balizas literárias e
cronológicas deste estudo.

No escopo dos discursos críticos a serem estudados, de 1888 a 1941, e
seus respectivos recortes do ``breve'' ou do ``longo'' século XIX, a
representação das cidades, tanto na pintura quanto na literatura,
desempenha um papel chave na construção do discurso sobre o óbito
prematuro de uma arte com caráter nacional e qualidade estética. Este
papel está centrado nas demarcações teóricas \emph{a priori} pelas quais
autores academicistas, ecléticos, tradicionalistas e modernistas de
variadas acepções aferem a falta de ``brasilidade'' da arte nacional. A
busca incessante por este caráter, na crítica do século XIX, é
indissociável das suas próprias exigências metodológicas que frustram,
necessariamente, essa mesma busca. Desse panorama, emergem posições
antagônicas aderentes não tanto a movimentos artísticos específicos,
muito menos à convencional contraposição entre tradição e vanguarda, mas
sobretudo a entendimentos emergentes acerca dos ofícios do crítico, do
artista e do arquiteto.

\begin{center}\rule{0.5\linewidth}{0.5pt}\end{center}

Alguns aspectos da ideologia da nação brasileira no século XIX têm sido
tratados de modo recorrente na produção científica atual: mormente, o
papel fulcral da Academia Imperial de Belas-Artes na constituição de um
cânone apto a gerar uma ``escola brasileira'' em termos admissíveis na
arte erudita europeia, tema minuciosamente estudado por Sônia Gomes
Pereira \autocite{pereira:2012revisao}, e o trânsito, frequentemente
contraditório, do ideal de nação entre esferas políticas e disciplinas
artísticas \autocite{abreu:2001museu}. Os papéis por vezes
intercambiáveis desempenhados por modernistas e tradicionalistas nas
campanhas pela caracterização da ``arte nacional'' constituem um
problema recorrente nos estudos sobre crítica de arte, mas também nas
áreas de história da literatura e da arquitetura
\autocite{lins:1996gonzaga25,thiengo:2010questao,wisnik:2007plastica}.

O estudo da cidade brasileira é pontuado por dois textos clássicos de
síntese publicados nos últimos cinquenta anos: \emph{Contribuição ao
estudo da evolução urbana do Brasil (1500--1720)}, de Nestor Goulart
Reis Filho \autocite*{reisfilho:1968contribuicao}, e \emph{Urbanismo no
Brasil, 1895--1965}, organizado por Maria Cristina da Silva Leme
\autocite*{leme:2005urbanismo}. Essas obras são representativas da
ênfase dada, ao longo deste meio século, aos períodos colonial e
republicano na pesquisa histórica sobre o espaço urbano brasileiro
\autocite{fernandes:2004historia}. Os lugares-comuns da estagnação
econômica subsequente ao ``ciclo do ouro'' e do conservadorismo
ruralista da sociedade imperial parecem ter desencorajado o estudo dessa
janela do desenvolvimento urbano brasileiro ao menos até o início deste
século, quanto despontam estudos sobre o incipiente mercado imobiliário
urbano \autocite{bueno:2005tecido} e sobre a morfologia dos traçados,
especialmente das cidades novas
\autocite{brazesilva:2012planejamento13,nogueira:2013analise}.

Essa distribuição evidencia uma ênfase, na história urbana brasileira,
na análise de projetos ou processos, preferencialmente intensos, de
crescimento das cidades, em detrimento de outros aspectos, tais como as
transformações na produção de discursos e imagens. Calca-se, também, na
esparsa ocorrência de discursos e projetos instauradores ou descritivos
do processo de urbanização produzidos no século XIX, à exceção daqueles
relativos à Corte. Neste contexto, as representações iconográficas e
literárias, segundo Pesavento \autocite*[ p.~665]{pesavento:2001busca},
constituem os ``vestígios esparsos {[}\ldots{]} dando a ver'' não apenas
o fato urbano, mas sobretudo os filtros ideológicos pelos quais este é
traduzido por suas testemunhas oculares.

Por sua vez, a reconstrução discursiva da cidade brasileira na primeira
República oculta, em parte, o \emph{locus} ideológico e estético da
vista urbana no entendimento do que seja a natureza da cidade
brasileira. De um lado, no Brasil imperial afirma-se um papel político
para a vista urbana na elaboração de uma ordem administrativa
centralista e hierarquizada --- evidenciada pela predominância de
panoramas de cidades administrativas e, dentro dessas, de palácios
governamentais ou igrejas associadas ao poder imperial. Do outro,
resgata-se a categoria estética do sublime
\autocite{naxara:2004cientificismo} como contraponto ao totalitarismo da
mensagem política. Nesse registro, a cidade brasileira é apropriada
enquanto intromissão de um ideal civilizador na onipotência da Natureza,
percebida, em última análise, como o elemento definidor do Brasil.
