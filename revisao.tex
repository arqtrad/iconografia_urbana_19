\hypertarget{periodizauxe7uxe3o-e-recortes-temuxe1ticos-um-diuxe1logo-necessuxe1rio}{%
\subsection{Periodização e recortes temáticos: um diálogo
necessário}\label{periodizauxe7uxe3o-e-recortes-temuxe1ticos-um-diuxe1logo-necessuxe1rio}}

A periodização da arte e, mais ainda, da arquitetura brasileira no
século XIX permanece carente de um marco de referência canônico. À
sombra dos estudos sobre a arte e a literatura barroca do século XVIII,
iniciados ainda sob o Império, e daqueles sobre o modernismo do século
XX, em muitos casos realizados pelos próprios protagonistas desse
movimento, o interesse pelo século XIX fica, por muito tempo, reduzido à
função de oferecer contraste aos períodos precedente e subsequente.
Mesmo no momento de revisão crítica dos cânones, na década de 1980, o
entendimento hegemônico da arte brasileira de Oitocentos será reafirmado
por \textcite{campofiorito:1983historia} e \textcite{fabris:1987luz}
enquanto reflexo imperfeito da produção europeia coetânea e enquanto
prenúncio incompleto da modernidade.

Na mesma época, a superação metodológica dos preconceitos modernistas no
bojo da profissionalização da história da arte e da arquitetura se
divide em duas abordagens bem distintas no tocante à periodização.
\textcite{pereira:1995mudanca1} representa o campo atualmente dominante
na história da arte e da literatura. Esse campo se caracteriza pela
recusa em atribuir significados metanarrativos à periodização,
preferindo recortes cronológicos explicitamente arbitrários. Já
\textcite{lemos:1989alvenaria} se reporta à autoridade de periodizações
canônicas em outras disciplinas, como a história econômica ou política.
Tal abordagem é representada num dos raros livros de síntese publicados
no século XXI, organizado segundo uma aderência total aos ``ciclos
econômicos'' convencionalmente adotados na história do Brasil
\autocite{bicca:2007arquitetura}. Ambas essas abordagens demonstram,
cada uma a seu modo, que o século XIX terá ``perdido o bonde'' das
periodizações próprias ao campo da história da arte e da arquitetura.

A ausência de uma periodização canônica para a arte e a arquitetura
brasileiras no século XIX implica o predomínio do recorte temático
monográfico sobre o recorte cronológico, onde este se torna consequência
daquele. Na prática, os objetos preferenciais da pesquisa recente se
distribuem na mesma clivagem entre dois campos disciplinares. Por um
lado, uma história predominantemente social da arte (compreendendo a
arquitetura) tem esquadrinhado a formação de conceitos
político-ideológicos calcados na periodização política convencional do
século XIX, tal como o proverbial programa de ``construção da
nacionalidade'' sob o Império \autocite{camargo:2012leitores1}. Por
outro, uma história da cidade (compreendendo aspectos
político-econômicos) tem avançado muito lentamente, desde os recortes
canônicos do ``ciclo do ouro'' e do ``ciclo'' da modernização
industrial, para ocupar o campo do século XIX.

Para a história da arte, destacam-se o papel fulcral da Academia
Imperial de Belas-Artes na constituição de um cânone apto a gerar uma
``escola brasileira'' em termos admissíveis na arte erudita europeia,
tema minuciosamente estudado por Sônia Gomes Pereira
\autocite*{pereira:2012revisao}, e o trânsito, frequentemente
contraditório, do ideal de nação entre esferas políticas e disciplinas
artísticas \autocite{abreu:2001museu}. Os papéis por vezes
intercambiáveis desempenhados por modernistas e tradicionalistas nas
campanhas pela caracterização da ``arte nacional'' constituem um
problema recorrente nos estudos sobre crítica de arte, mas também nas
áreas de história da literatura e da arquitetura
\autocite{lins:1996gonzaga25,thiengo:2010questao,wisnik:2007plastica}.

O estudo da cidade brasileira, por sua vez, é pontuado por dois textos
clássicos de síntese publicados nos últimos cinquenta anos:
\emph{Contribuição ao estudo da evolução urbana do Brasil (1500--1720)},
de Nestor Goulart Reis Filho \autocite*{reisfilho:1968contribuicao}, e
\emph{Urbanismo no Brasil, 1895--1965}, organizado por Maria Cristina da
Silva Leme \autocite*{leme:2005urbanismo}. Essas obras são
representativas da ênfase dada, ao longo deste meio século, aos períodos
colonial e republicano na pesquisa histórica sobre o espaço urbano
brasileiro \autocite{fernandes:2004historia}. Os lugares-comuns da
estagnação econômica subsequente ao ``ciclo do ouro'' e do
conservadorismo ruralista da sociedade imperial parecem ter
desencorajado o estudo dessa janela do desenvolvimento urbano brasileiro
ao menos até o recentemente, quanto despontam estudos sobre o incipiente
mercado imobiliário urbano \autocite{bueno:2005tecido} e sobre a
morfologia dos traçados, especialmente das cidades novas
\autocite{brazesilva:2012planejamento13,nogueira:2013analise}.

Essa distribuição evidencia um viés, na história urbana brasileira, para
a análise de projetos ou processos, preferencialmente intensos, de
crescimento das cidades, em detrimento de outros aspectos, tais como as
transformações na produção de discursos e imagens. Calca-se, também, na
esparsa ocorrência de discursos e projetos instauradores ou descritivos
do processo de urbanização produzidos no século XIX, exceção feita
daqueles relativos à Corte. Neste contexto, as representações
iconográficas e literárias, segundo Pesavento
\autocite*[p.~665]{pesavento:2001busca}, constituem os ``vestígios
esparsos {[}\ldots{]} dando a ver'' não apenas o fato urbano, mas
sobretudo os filtros ideológicos pelos quais este é traduzido por suas
testemunhas oculares.

Por sua vez, a reconstrução discursiva da cidade brasileira na primeira
República sobrepuja, em parte, o \emph{locus} ideológico e estético da
vista urbana no entendimento do que seja a natureza da cidade brasileira
durante o período imperial. Todavia, o interesse temático na vista
urbana como expressão de ideologias, e especialmente da construção
antropológica da nacionalidade, convida a abarcar ambos os regimes
políticos. De um lado, no Brasil imperial afirma-se um papel político
para a vista urbana na elaboração de uma ordem administrativa
centralista e hierarquizada --- evidenciada pela predominância de
panoramas de cidades administrativas e, dentro dessas, de palácios
governamentais ou igrejas associadas ao poder imperial
\autocite{barata:1996alguns}. Do outro, resgata-se a categoria estética
do sublime \autocite{naxara:2004cientificismo} como contraponto ao
totalitarismo da mensagem política. Nesse registro, a cidade brasileira
é apropriada enquanto intromissão de um ideal civilizador na onipotência
da Natureza, percebida, em última análise, como o elemento definidor do
Brasil.
