Alguns aspectos da ideologia da nação brasileira no século XIX têm sido
tratados de modo recorrente na produção científica atual: mormente, o
papel fulcral da Academia Imperial de Belas-Artes na constituição de um
cânone apto a gerar uma ``escola brasileira'' em termos admissíveis na
arte erudita europeia, tema minuciosamente estudado por Sônia Gomes
Pereira \autocite{pereira:2012revisao}, e o trânsito, frequentemente
contraditório, do ideal de nação entre esferas políticas e disciplinas
artísticas \autocite{abreu:2001museu}. Os papéis por vezes
intercambiáveis desempenhados por modernistas e tradicionalistas nas
campanhas pela caracterização da ``arte nacional'' constituem um
problema recorrente nos estudos sobre crítica de arte, mas também nas
áreas de história da literatura e da arquitetura
\autocite{lins:1996gonzaga25,thiengo:2010questao,wisnik:2007plastica}.

O estudo da cidade brasileira é pontuado por dois textos clássicos de
síntese publicados nos últimos cinquenta anos: \emph{Contribuição ao
estudo da evolução urbana do Brasil (1500--1720)}, de Nestor Goulart
Reis Filho \autocite*{reisfilho:1968contribuicao}, e \emph{Urbanismo no
Brasil, 1895--1965}, organizado por Maria Cristina da Silva Leme
\autocite*{leme:2005urbanismo}. Essas obras são representativas da
ênfase dada, ao longo deste meio século, aos períodos colonial e
republicano na pesquisa histórica sobre o espaço urbano brasileiro
\autocite{fernandes:2004historia}. Os lugares-comuns da estagnação
econômica subsequente ao ``ciclo do ouro'' e do conservadorismo
ruralista da sociedade imperial parecem ter desencorajado o estudo dessa
janela do desenvolvimento urbano brasileiro ao menos até o início deste
século, quanto despontam estudos sobre o incipiente mercado imobiliário
urbano \autocite{bueno:2005tecido} e sobre a morfologia dos traçados,
especialmente das cidades novas
\autocite{brazesilva:2012planejamento13,nogueira:2013analise}.

Essa distribuição evidencia uma ênfase, na história urbana brasileira,
na análise de projetos ou processos, preferencialmente intensos, de
crescimento das cidades, em detrimento de outros aspectos, tais como as
transformações na produção de discursos e imagens. Calca-se, também, na
esparsa ocorrência de discursos e projetos instauradores ou descritivos
do processo de urbanização produzidos no século XIX, à exceção daqueles
relativos à Corte. Neste contexto, as representações iconográficas e
literárias, segundo Pesavento \autocite*[ p.~665]{pesavento:2001busca},
constituem os ``vestígios esparsos {[}\ldots{]} dando a ver'' não apenas
o fato urbano, mas sobretudo os filtros ideológicos pelos quais este é
traduzido por suas testemunhas oculares.

Por sua vez, a reconstrução discursiva da cidade brasileira na primeira
República oculta, em parte, o \emph{locus} ideológico e estético da
vista urbana no entendimento do que seja a natureza da cidade
brasileira. De um lado, no Brasil imperial afirma-se um papel político
para a vista urbana na elaboração de uma ordem administrativa
centralista e hierarquizada --- evidenciada pela predominância de
panoramas de cidades administrativas e, dentro dessas, de palácios
governamentais ou igrejas associadas ao poder imperial. Do outro,
resgata-se a categoria estética do sublime
\autocite{naxara:2004cientificismo} como contraponto ao totalitarismo da
mensagem política. Nesse registro, a cidade brasileira é apropriada
enquanto intromissão de um ideal civilizador na onipotência da Natureza,
percebida, em última análise, como o elemento definidor do Brasil.
